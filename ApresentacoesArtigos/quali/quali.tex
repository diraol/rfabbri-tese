\documentclass[a4paper,openright,12pt]{report} %,parskip=full

\usepackage{amssymb,amsmath,textcomp}
\usepackage[brazil]{babel}
\usepackage[utf8]{inputenc}
\usepackage{braket}
\usepackage{booktabs}  %Possibilidade de linhas mais grossas nas tabelas.

%MEU
%\usepackage[brazilian, portuguese, activeacute]{babel}
%\usepackage{graphicx}
\usepackage{wrapfig}
\usepackage{subfigure}
%\usepackage{color}
%\usepackage{amssymb}
\usepackage{amsmath}  %for binom, not
%\usepackage{pifont}   %for ding
\usepackage{hyperref}
%\usepackage{amsthm}   %for what
%\usepackage{helvet}   %for what
\usepackage{cancel}   %for cancel
\hypersetup{
    colorlinks,
    citecolor=black,
    filecolor=black,
    linkcolor=black,
    urlcolor=black
}

\usepackage{etex}

\usepackage{cite}



%\usepackage[footnotesize,hang]{caption} 

%\usepackage[hang,small,labelsep=endash]{caption} % Hifen na legenda de tableas e figuras
\usepackage[hang,footnotesize,labelsep=endash,tableposition=top]{caption} % Hifen na legenda de tableas e figuras
%CHCHCH
%\usepackage{subcaption}

%%%%%%%%%%%%%%%%% Bibliografia
\usepackage{url}
\usepackage[bibjustif,abnt-thesis-year=both,num]{abntcite} %%%%%% ABNY
%%%%%%%%%%%%%%%%% Espacamento
%\makeatletter  %Espa�amento entre os itens da bibliografia
%\newcommand{\adjustmybblparameters}{\setlength{\itemsep}{2\baselineskip}\setlength{\parsep}{0.5ex}}
%\let\ORIGINALlatex@openbib@code=\@openbib@code  
%\renewcommand{\@openbib@code}{\ORIGINALlatex@openbib@code\adjustmybblparameters}
%\makeatother
%%%%%%%%%%%%%%%%%%%%%%%%%%%%%%%%%%%
% Espacaçamento entre as referências
\makeatletter  %Espa�amento entre os itens da bibliografia
\newcommand{\adjustmybblparameters}{\setlength{\itemsep}{2\baselineskip}\setlength{\parsep}{0.5ex}}
\let\ORIGINALlatex@openbib@code=\@openbib@code  
\renewcommand{\@openbib@code}{\ORIGINALlatex@openbib@code\adjustmybblparameters}
\makeatother

%%%%%%%%%%%%%%%%%%%%%%%%%%%%%%%%%%%%%%%%%%%%%%%
%%%%%%%%%%%%%%%%%%%%%%%%%%%Symbol footnote
\long\def\symbolfootnote[#1]#2{\begingroup%
\def\thefootnote{\fnsymbol{footnote}}\footnote[#1]{#2}\endgroup}
%%%%%%%%%%%%%%%%%%%%%%%%%%

\usepackage[Conny]{fncychap}
\usepackage{appendix}

%%%%%%%%%%%%%%%%%%%%%%%%%%%%%%%%%%%%%%%5

\makeatletter
\renewcommand*{\@makechapterhead}[1]{%
  \vspace*{-1.0cm}%
  {\parindent \z@ \raggedright \normalfont
    \ifnum \c@secnumdepth >\m@ne
      \if@mainmatter%%%%% Fix for frontmatter, mainmatter, and backmatter 040920
        \DOCH
      \fi
    \fi
    \interlinepenalty\@M
    \if@mainmatter%%%%% Fix for frontmatter, mainmatter, and backmatter 060424
      \DOTI{#1}%
    \else%
      \DOTIS{#1}%
    \fi
  }}

%older
%%%%%%%% Espaçamento entre títulos de seções e linhas anteriores e subsequentes
%\usepackage[compact]{titlesec}
%\titlespacing{\section}{0pt}{0.5cm}{0.5cm}
%%\titlespacing{\chapter}{0pt}{2.0cm}{2.0cm}
%\titlespacing{\subsection}{0pt}{0.5cm}{0.5cm}
%\titlespacing{\subsubsection}{0pt}{0.5cm}{0.5cm}	

%%abnt mestrado
%%%%%%%%% Espaçamento entre títulos de seções e linhas anteriores e subsequentes
%\usepackage{titlesec} %[compact]
%\titlespacing{\chapter}{0pt}{1.9cm}{2.2cm}
%\titlespacing{\section}{0pt}{1.9cm}{2.2cm}
%\titlespacing{\subsection}{0pt}{2.1cm}{2.2cm}
%\titlespacing{\subsubsection}{0pt}{2.2cm}{2.2cm}	

%quali
%%%%%%%% Espaçamento entre títulos de seções e linhas anteriores e subsequentes
\usepackage{titlesec} %[compact]
\titlespacing{\chapter}{0pt}{0cm}{1.5cm}
\titlespacing{\section}{0pt}{1.5cm}{1.5cm}
\titlespacing{\subsection}{0pt}{1.5cm}{1.5cm}
\titlespacing{\subsubsection}{0pt}{1.5cm}{1.5cm}	

%  \usepackage{titlesec}
  \titleformat{\chapter}[display]
  {\normalfont\bfseries\filcenter}
  {\LARGE\thechapter}
  {-2ex}
  {%\titlerule[2pt]
	 % \vspace{2ex}%
  \LARGE}
  [%\vspace{1ex}%
  {\titlerule[1pt]}]


%%%%%%%%%% Escreve ``Figura`` e ``Tabela`` nas listas de figuras e tabelas
%%CHCHCH
%\usepackage[titles]{tocloft}
%\renewcommand{\cftfigaftersnum}{ -}
%\renewcommand{\cfttabaftersnum}{ -}
%\renewcommand{\cftfigpresnum}{Figura }
%\renewcommand{\cfttabpresnum}{Tabela }
%\setlength{\cftfignumwidth}{2.2cm}
%\setlength{\cfttabnumwidth}{2.2cm}
%%%%%%%%%%%%%%%%%%%%%%%%%%%%%%%%


\usepackage[pdftex]{graphicx}  %pacote para colocar figuras jpg
\usepackage[final]{pdfpages}  %include pfd coloca um pdf em uma pagina

%\usepackage[T1]{fontenc} %CHCHCH
%\usepackage[OT2,T1]{fontenc} %CHCHCH
%\usepackage[T1,T2A]{fontenc} %needs t2aenc.def
\usepackage[T1,OT2,OT1]{fontenc} %CHCHCH

%\usepackage{cyrillic} 

%russian
\newcommand\cyr{%
\renewcommand\rmdefault{wncyr}%
\renewcommand\sfdefault{wncyss}%
\renewcommand\encodingdefault{OT2}%
\normalfont
\selectfont}
\DeclareTextFontCommand{\textcyr}{\cyr}

%%CHCHCH
%%russian
%\newcommand{\cyrrm}{\fontencoding{OT2}\selectfont\textcyrup}
%\newcommand{\cyrit}{\fontencoding{OT2}\selectfont\textcyrit}
%\newcommand{\cyrsl}{\fontencoding{OT2}\selectfont\textcyrsl}
%\newcommand{\cyrsf}{\fontencoding{OT2}\selectfont\textcyrsf}
%\newcommand{\cyrbf}{\fontencoding{OT2}\selectfont\textcyrbf}
%\newcommand{\cyrsc}{\fontencoding{OT2}\selectfont\textcyrsc}
%%%%% cyrrm = "Roman", or really upright, normal font
%%%%% cyrit = Italic (cursive forms of letters)
%%%%% cyrsl = Italic (non-cursive forms of letters)
%%%%% cyrsf = Sans-serif
%%%%% cyrbf = Bold-face 




%\usepackage[math]{iwona}
\renewcommand{\familydefault}{\sfdefault}
%\renewcommand{\familydefault}{\sfdefault} math

\usepackage[absolute]{textpos}
\usepackage[retainorgcmds]{IEEEtrantools}
\usepackage{leftidx}

%CHCHCH
%\usepackage[superscript]{cite}  %citacoes em superscript
\usepackage{cite}  %nao, meu, citacoes em superscript nao!

\usepackage{tikz}
\usetikzlibrary{arrows}
\usetikzlibrary{decorations}
\usetikzlibrary{snakes}

\usepackage{color}

\definecolor{DarkBlue}{rgb}{0.1,0.1,0.5}
\definecolor{Red}{rgb}{0.9,0.0,0.1}
\definecolor{DarkGreen}{rgb}{0.10,0.50,0.10}

% see documentation for a0poster class for the size options here
\let\Textsize\large
\def\Head#1{\noindent\hbox to \hsize{\hfil{\LARGE\color{DarkBlue} #1}}\bigskip}
\def\LHead#1{\noindent{\LARGE\color{DarkBlue} #1}\smallskip}
\def\Aut#1{\noindent{\Huge\color{DarkBlue} #1}\smallskip}
\def\End#1{\noindent{\large\it\color{DarkBlue} #1}\smallskip}
\def\Subhead#1{\noindent{\Large\color{DarkBlue} #1}}
\newcommand{\quiteHuge}{\fontsize{120}{93}\selectfont}
\def\Title#1{\begin{center}\noindent{\quiteHuge\color{DarkGreen}#1}\end{center}}


\usepackage[twoside,inner=3cm,outer=2cm,top=3cm,bottom=2cm]{geometry}

\usepackage{setspace}   %Espacamento entre as linhas

\setstretch{1.5} %Normal da ABNT: espaçamento entre as linhas de 1.5 de linha
% Pode ser \singlespace 
%          \onehalfspace
%          \doublespace

%---------------------------------------------------------------------------------------------------------
\usepackage{fancyhdr}
\pagestyle{fancy} % colocar Capítulos, Seções, etc em minúsculo

\renewcommand{\sectionmark}[1]{\markright{\thesection\ #1}}

\fancyhf{} % deletar configuração atual do cabeçalho (header) e rodapé (foot)

\pagestyle{fancy} 

\fancyhead[LE,RO]{\thepage}

\fancyhead[LO]{\rightmark}
  
\fancyhead[RE]{\leftmark}

\renewcommand{\headrulewidth}{0.5pt}
  
\renewcommand{\footrulewidth}{0pt}

%CHCHCH
%\addtolength{\headheight}{0.5pt} % cria um espaço para linha
\addtolength{\headheight}{4.0pt} % cria um espaço para linha

\fancypagestyle{plain}{
    \fancyhead{} % exibir cabeçalho e rodapé
    \renewcommand{\headrulewidth}{0pt} % linha
}




%---------------------------------------------------------------------------------------------------------

\usepackage[marginal,symbol]{footmisc}
\footnotemargin2pt

%%%%%%%%%%%%%%%%%%


\pagestyle{empty}

%\AtBeginDocument{\addtocontents{toc}{\protect\thispagestyle{empty}}} %CHCHCH %to make tableofcontents empty pagestyle


%CHCHCH TO MAKE TABLEOFCONTENTS PAGESTULE EMPTY
%\fancypagestyle{plain}{%
%  \fancyhf{}                          % clear all header and footer fields
%  \renewcommand{\headrulewidth}{0pt}
%  \renewcommand{\footrulewidth}{0pt}
%}

\usepackage[subfigure,titles]{tocloft}
\renewcommand{\cftfigaftersnum}{ -}
\renewcommand{\cfttabaftersnum}{ -}
\renewcommand{\cftfigpresnum}{Figura }
\setlength{\cftfignumwidth}{2.2cm}%{5em}



%\renewcommand*{\chapterheadendvskip}{%
%  \vspace{3.0cm}%
%%  \vspace{0.725\baselineskip plus 0.115\baselineskip minus 0.192\baselineskip}%
%}

\usepackage{blindtext}

%%%%%%%%%%%%%%%%%%%%%%%%%%%%%%%%%%%%%%%%%%%%%%%%%%%%%%%%%%%%%%%%%%%%%%%%%%%%%%%%%%%%%%%%%%%%%%%%%%%%%%%%%%%%%%%%%%%%%%%%%%%%%%%%%%%%%%%%%%%%%%%%%%%
%%%%%%%%%%%%%%%%%%%%%%%%%%%%%%%%%%%%%%%%%%%%%%%%%%%%%%%%%%%%%%%%%%%%%%%%%%%%%%%%%%%%%%%%%%%%%%%%%%%%%%%%%%%%%%%%%%%%%%%%%%%%%%%%%%%%%%%%%%%%%%%%%%%

\begin{document}

%\setlength{\parskip}{0.5cm}%1.5ex} %0pt} % 1ex plus 0.5ex minus 0.2ex}

\hyphenation{ca-ra-te-ri-sti-cas}

%\selectlanguage{portuguese}
%\hyphenation{e-mer-g\^{e}n-cia}


\pagestyle{fancy}


\thispagestyle{empty}

\vspace{0.5cm}

\begin{center} 
%\LARGE{UNIVERSITY OF SÃO PAULO}  \\
%\LARGE{SÃO CARLOS PHYSICS INSTITUTE}
\LARGE{UNIVERSIDADE DE SÃO PAULO}  \\
\LARGE{INSTITUTO DE FÍSICA DE SÃO CARLOS}
\end{center}

\vspace{6.0cm}

\centerline{\LARGE{RENATO FABBRI}}

\vspace{3.0cm}


%\centerline{\Huge{Complex networks for the participant}} 
\centerline{\Huge{Redes complexas para o participante}} 
\vspace{0.5cm}
%\centerline{\Huge{aplicada a reconhecimento de autoria}}


\vspace{6.5cm}

\begin{center}
\Large{S\~ao Carlos}\\
\Large{2015}
\end{center}



\newpage\ \thispagestyle{empty}  \newpage\thispagestyle{empty}

\setcounter{page}{1} % ABNT: deve-se começar a numeração na folha de rosto.

\begin{center}
\LARGE{RENATO FABBRI}
\end{center}

\addvspace{4.0cm}

\begin{center}
%\Huge{Complex networks for the participant}
\Huge{Redes complexas para o participante}
\end{center}

\addvspace{3.0cm}

\makebox[15cm][r]{
\begin{minipage}[l]{8cm}

\begin{singlespace}
Monografia apresentada ao Programa de Pós-Graduação em Física 
do Instituto de Física de São Carlos da Universidade de São Paulo, para o Exame de Qualifica\c c\~ao 
como parte dos requisitos para obten\c c\~ao do t\'itulo de Doutor em Ci\^encias.\\

Área de concentração: Física Aplicada

Opção: Física Computacional

Orientador: Prof. Dr. Osvaldo Novais de Oliveira Jr.
%Monograph presented to the Physics Graduate Program of the
%São Carlos Institute of Physics, University of São Paulo,
%for the qualification exam, as part of the requirements
%for obtaining the title of Doctor in Sciences.\\
%
%Concentration area: Applied Physics
%
%Option: Computational Physics
%
%Advisor: Prof. Dr. Osvaldo Novais de Oliveira Jr.
\end{singlespace}


\end{minipage}}

\addvspace{3.0cm}

\begin{center}
%\Large{Versão original} \\
\vspace{1.5cm}
\Large{S\~ao Carlos}\\
\Large{2015}
\end{center}


%%%%%%
%\thispagestyle{empty}
%
%
%\includepdf{index}
%\includepdf{fichacatalog135}
%%%%%%

%% \vspace*{8cm}
%%\begin{center}
%% \huge{FICHA CATALOGRÁFICA}

%% \vspace*{3cm}

%%\large{Elaborar ao final, quando o número de páginas da 
%%dissertação estiver definido - 
%%www.biblioteca.ifsc.usp.br/ficha}
%%\end{center}

%%%%%%%%%%%
%\newpage\thispagestyle{empty} 
%
%\vspace*{8cm}
%
%\begin{center}
%\normalsize{FOLHA DE APROVAÇÃO}
%\end{center}
%
%
%
%\newpage\ \thispagestyle{empty}  \newpage\thispagestyle{empty}\ 
%
%\vspace*{18cm}
%
%\begin{spacing}{1.2}
%
%\begin{flushright}\textit{
%\`A MINHA AV\'O,\\
%\vspace{0.5cm}
%CLELIA C\'ESPEDES ACERO
%\vspace{1cm}
%}\end{flushright}
%
%\end{spacing}
%%%%%%%%%%%



%%%%%%%%%%%%%%%%%%%%
%\newpage\ \thispagestyle{empty} %NEWW
%
%\newpage\ \thispagestyle{empty}
%
%
%\vspace*{18cm}
%
%\begin{spacing}{1.2}
%
%\makebox[15cm][r]{ 
%\begin{minipage}[l]{10cm}
%\hspace*{2.2cm}\emph{ \large{ Este trabalho foi financiado pelo \\ \hspace*{7.5cm} CNPq } }
%\end{minipage}}
%
%
%
%\end{spacing}
%%%%%%%%%%%%%%%%%%%%





%\newpage\ \thispagestyle{empty}  \newpage\thispagestyle{empty}

\newpage\ \thispagestyle{empty}  \newpage\thispagestyle{empty}


\begin{singlespace}

%\centerline{\LARGE{{\bf ABSTRACT}}}
\centerline{\LARGE{{\bf RESUMO}}}

\vspace*{1.5cm}

%\hspace*{-0.9cm} FABBRI, C. \textit{Complex networks for the participant}. 
\hspace*{-0.9cm} FABBRI, C. \textit{Redes complexas para o participante}. 
Instituto de Física de São Carlos, Universidade de São Paulo, São Carlos, 2015. 
%São Carlos Institute of Physics. University of São Paulo, São Carlos, 2015. 

\vspace*{1.2cm}

\hspace*{-0.9cm}

{\noindent
	As redes complexas formam uma das áreas mais ativas da física redente.
	Há esforços consideráveis para apresentar estes avanços ao público geral, mas tudo indica
	que poucos ou nenhum são voltados para o benefício do indivíduo que constitui estes sistemas.
	Ou seja, com um núcleo de conhecimento da área, e receitas para aproveitamento,
	fornece meios para o participante interagir e entender as redes nas quais ele se encontra.
	Este trabalho objetiva realizar tal tarefa por meio das redes sociais do participante.
	Verificamos que tais redes exibem uma estabilidade temporal de medidas temporais e dos
	tamanhos relativos dos setores conectivos básicos (hubs, intermediários, periféricos).
	Observamos uma diferenciação da produção de texto de cada setor básico.
	Também formalizamos as conceitualizações destas redes como OWL onde foi possível,
	principalmente as instâncias de participação dosial previstas por lei e implementadas como software.
	Software e dados foram disponibilizados e usados. Protocolos escolhidos para
	facilitar a integração de estruturas de diferentes procedências,
	para reutilização dos dados em outros trabalhos e pesquisas, e para o benefício público.
	Consequências conceituais requerem considerações antropológocas e estão sendo documentadas.
	Próximos passos são: melhor documentação e desenvolvimento do aparato em software, ontologias e dados;
	considerações tipológicas das propriedades físicas observadas nas redes de interação humana, com
	atenção aos outliers, às relações entre topologia do agente e texto produzido, e à ponte com
	a bagagem mais tradicional das ciências humanas no assunto.
}

%{\noindent
%	Complex networks form one of the most active fields of recent physics.
%	With respectable efforts for exhibiting advances to the general audience, 
%	it seems, however, that few or none of these are targeted to the benefit of the individual that constitute such systems.
%	That is, with a core knowledge about the field, and recipes for harnessing,
%	provide means for the participant to interact and understand the networks he/she is in.
%	This work aims to accomplish such task by means of the social networks of the participants.
%	We verified that such networks exhibit time stability of topological measures and
%	of basic connective sector sizes and exhibit differentiation of the textual production in each basic connective sector.
%	We also formalized conceptualizations of these networks as OWL were they were possible, specially
%	in relation to the social participation instances provided by law. Finally, software and
%	data have been put available and used, as means to enable integrated analysis of different provenance and public benefit.
%	Conceptual consequences have been documented and requires anthropological considerations.
%	Furthermore, software, ontological and data contributions can be better documented and developed while
%	a typological consideration of the physical properties observed in human interaction networks should
%	bridge complex networks and the more traditional legacy of human sciences on the subject.
%}

\vspace*{1.5cm}
%\hspace*{-0.9cm} {\bf Keywords:} Complex networks. Social networks. Complexity. Anthropological physics. Linked data. Semantic web. Social participation. Text mining. Natural language processing.
\hspace*{-0.9cm} {\bf Palavras-chave:} Redes complexas. Redes sociais. Complexidade. Física antropológica.
Dados ligados. Web semântica. Participação social. Mineração de texto. Processamento de linguagem natural.

\end{singlespace}

  

%%%%%%%%%%%%%%%%%%%%%%%%%%%%%
%\newpage\thispagestyle{empty}
%
%\pagestyle{empty}
%
%\begin{singlespace}
%\listoffigures
%%\listoftables
%\end{singlespace}
%%%%%%%%%%%%%%%%%%%%%%%%%%%%%

%%%
%\newpage\ \thispagestyle{empty}  \newpage\thispagestyle{empty}

\newpage\ \thispagestyle{empty}  \newpage\thispagestyle{empty}
%\thispagestyle{empty} %CHCHCH

\tableofcontents\thispagestyle{empty}\thispagestyle{empty}%\thispagestyle{empty}

\clearpage \thispagestyle{empty}

\pagestyle{fancy}

%\chapter{Introduction}
\chapter{Introdução}
Estudos sobre redes de interação humana foram iniciados bem antes dos computadores modernos,
datam do século XIX, enquanto a fundação da ``Análise de Redes Sociais''/ARS (\emph{Social Network Analysis}/SNA)
é geralmente atribuida ao psiquiatra Jacob Moreno na metade do século vinte~\cite{newmanBook}.
Com a crescente disponibilidade de dados relacionados à interação humana, a pesquisa destas redes tem aumentado continuamente.
Contribuições podem ser encontradas em uma variedade de áreas, de ciências sociais e humanidades~\cite{latour2013}
a ciências sociais~\cite{bird} e física~\cite{barabasiHumanDyn,newmanFriendship},
dada a natureza multidisciplinar do assunto.
Uma das abordagens da perspectiva de uma ciência exata é representar a rede de interação como uma rede
complexa~\cite{barabasiHumanDyn,newmanFriendship},
com a qual algumas características foram reveladas.
Por exemplo, a topologia das redes de interação humana exibem um traço livre de escala,
o que aponta para a existência de um pequeno número de hubs super conectados
e um grande número de vértices pouco conectados.

Há um hiato de conhecimento e tecnologi entre o legado de redes complexas e o usufruto do participante.
Este hiato é reativo, e há evidência de que conseguirá se manter como um ecossistema de conhecimento, tecnologia
e empreendimento da sociedade em todas as suas escalas.
Deve facilitar, por exemplo:
elaboração e preparação de documentos, aquisição rápida de conhecimento,
realização de empreitadas coletivas.
Em geral: processos de coleta e difusão de informação (e bens).
% Argumentar que é tido que o correto entendimento destas estruturas
% será mais forte que XXX (ver no curso da Melanie/CE).

Este trabalho apresenta uma confirmação deste cenário e avanços.
Algumas estratégias foram selecionadas para verificar a aplicabilidade
de conceitos de redes complexas para o benefício do participane.
Em especial, experimentos muito simples parecem capazes de modificar
estruturas sociais. Neste contexto, verificamos estabilidades temporais
nas redes de interação humana, e expomos que os setores primitivos
das redes (hubs, intermediários e periféricos)
produzem textos bastante diferentes entre si.
Este conhecimento é útil para uma tipologia não estigmatizante
de participantes em redes de interação.
A audiovisualização e interconexão de dados com arte e engenhocas
em software deram suporte contínuo à pesquisa científica.
Aplições foram complementadas com a Presidência da República
e o PNUD/ONU.

A próxima seção apresenta considerações gerais sobre a literatura.
A Seção~\ref{sec:mat} é dedicada aos dados analizados.
A Seção~\ref{sec:met} contém os métodos usados para atingir os resultados, que são explicitados na Seção~\ref{sec:res}. O cronograma de atividades e uma comparação entre afazeres planejados, em andamento e finalizados
estão na Seção~\ref{sec:chr}.
A monografia termina com as conclusões na Seção~\ref{sec:con}, seguida de agradecimentos e bibliografia.

% Interacting with own network, with constant feedback, study of the self
% Annotation of own data, study of them. Extend analysis within the route.
%In this work, we present advances in handing the participant ways to explore and harness the social networks he/she participates in.
%This is not an easy task, therefore we selected

\section{Revisão de literatura}
A área das redes complexas é relativamente nova ($\approx 25$ anos)
e a literatura apresenta definições divergentes da área em si.
Uma definição que tem recebido aceitação crescente
é da rede complexa como ``um grafo grande com características topológicas
não triviais''. Esta definição é enganosa ao menos em três pontos.
Primeiro, há redes de interesse com características topológicas triviais, como as redes de Erdös-Rényi e a Geográfica~\cite{newman}, ou as redes simples usadas para exemplos.
Segundo, a definição falha ao não emitir a mensagem fundamental de que uma rede complexas não é somente uma estrutura matemática, um grafo isolado. As redes complexas de interesse são redes reais ou modelos idealizados para as entender.
Além disso, não só grafos grandes são de interesse, mas grafos pequenos são comumente usados como exemplos de propriedades e extensão das estruturas maiores.
Uma definição, ainda longe de perfeita, mas preferida neste trabalho, é considerar a área das redes complexas como interessada em
``redes usualmente grandes, consideradas no, ou para consideração do,
meio em que residem''.
This definition resolves both issues.

Os livros em geral apresentam um comum e poderoso repertório para a caracterização de sistemas complexos através de grafos. Talvez mais notáveis sejam:
\begin{itemize}
	\item O arsenal de medidas: grau, força, betweenness centrality, coeficiente de clusterização, etc.
	\item Os paradigmas básicos de redes: Erdös-Rényi, geográfica, de mundo pequeno e livre de escala.
	\item A abordagem transdisciplinar para considerar o meio no qual a rede está inserida, ou que implica na rede.
\end{itemize}

A literatura sobre análise de redes sociais (ARS, ou \emph{SNA} para \emph{Social Network Analysis}), por exemplo, pode ser frequentemente compreendida como redes complexas em sistemas sociais humanos.

Uma consideração cuidadosa dos livros e artigos lidos para esta pesquisa estão na Seção~\ref{}. As seções a seguir (\ref{sec:jar} e~\ref{sec:misc})
explicitam peculiaridades do jargão da área e considerações sobre as áreas secundárias.

\subsection{Ambiguidades e sinonimos no jargão}\label{sec:jar}
A área de redes complexas é recente e conflui com diversas 
correntes científicas, como a física, a biologia e a sociologia.
Assim, possui termos ambiguos e sinònimos.

Exemplos de ambiguidade e delimitações adotadas:
\begin{itemize}
	\item Os vértices mais conectados são, por definição, chamados hubs da rede. O vértice mais conectado é chamado hub da rede. No contexto do algoritmo HITS, o que é bem comum, estes significados mudam: os hubs são os que possuem mais arestas saindo (grau de saída); as autoridades recebem as arestas, ou são referenciados por vários hubs e outras entidades.
	\item Há uma definição de centro e periferia com relação ao raio e diâmetro da rede~\cite{newman,networkX}.
		Por extensão os intermediários podem ser considerados os que não são centro nem periferia.
		Esta setorialização centro, intermediários e periferia gera frações que diferem do previsto pela literatura para as frações de hubs, intermediários e periféricos.
		Um método apropriado para realizar esta setorialização da rede, com resultados estáveis e significativos, consta na Seção~\ref{sec:set}.
	\item etc
\end{itemize}

\subsection{Processamento de linguagem natural, dados ligados, participação social}\label{sec:misc}
Os termos processamento de linguagem natural (PLN) e mineração de
texto (MT) podem em geral serem substituídos um pelo outro.
O termo PLN é preferido pois os intuitos da pesquisa são muito mais
próximos aos intuitos da área: compreender como a linguagem verbal está
sendo usada para significar.

Os termos web semântica e dados ligados em geral também podem ser 
substituídos um pelo outro. O primeiro salienta
a rede de referenciamento dos dados,
o segundo os dados referenciando-se. Principalmente na esfera
acadêmica, a área é, salvo segunda ordem, sinônimo de dados em 
RDF via XML ou Turtle, ontologias OWL e máquinas de inferència.

A participação social é a incorporação da própria sociedade nos 
processos de governança da sociedade.
Quase toda a participação social atual é indireta e presencial, 
com a população fornecendo diretrizes e indicadores para o setor público.
A transparência tem sido cada vez mais presente,
e o norte de ``participação direta'' (participação direta 
da sociedade civil na tomada de decisões pelo Estado)
cada vez mais presente. 
% discursar sobre a literatura de cada tema.

%\chapter{Materials}\label{sec:mat}
\chapter{Materiais}\label{sec:mat}
%\section{The Gmane public database of email lists (benchmarks)}
\section{O banco Gmane de dados públicos sobre listas de email (benchmark)}

Mensagens de lista de email foram obtidas do arquivo Gmane~\cite{gmanePack},
que consiste em mais de 20 mil listas de email e mais de 130 milhões de mensagens~\cite{GMANEwikipedia}.
Estas listas cobrem uma variedade de assuntos, em especial
relacionados à tecnologia. O arquivo pode ser descrito como
um corpus com metadados de emails, que incluem hora e lugar
de envio, nome e email do remetente. O uso do GMANE para
pesquisa científica é incidente no estudo de listas isoladas
e de inovações lexicais~\cite{GMANE2,bird}. 

\section{Facebook, Twitter, Participa.br, Cidade Democrática, AA}
Embora as redes de email tenham sido usadas como referência
na observação de propriedades gerais, outras fontes
foram analisadas:

\begin{itemize}
	\item Redes de amizade e interação do Facebook. 8 são usadas como referência em~\cite{timeS}, mas dezenas, talvez algumas centenas, foram observadas nos experimentos da Seção~\ref{sec:exp}.
	\item Milhares de tweets (talvez alguns milhões), geralmente vinculados à alguma \emph{hashtag}. Em especial, a rede de retweets de 22 mil tweets com a hashtag \#arenaNETmundial, foi analizada em~\cite{timeS}.
	\item Mecanismos participativos como o Participa.br, Cidade Democrática e o AA. As redes de amizade e de interação do Participa.br foram analizadas em~\cite{timeS}.
\end{itemize}

\chapter{Métodos}
Para realização desta pesquisa, foram
necessários métodos consagrados, adequações
e variantes.
Esta seção expõe uma seleção destes métodos,
para organizar o conhecimento
e exemplificar esta diversidade:
\begin{itemize}
	\item A Seção~\ref{sec:cir} expõe medidas simples
de estatística circular, ou direcional.
A contribuição neste caso é unicamente nos
padrões encontrados, o método é bastante estabelecido.
	\item A Seção~\ref{sec:intNet} expõe a síntese de redes
		de interação. Talvez haja contribuição na síntese
		do conceito de redes de interação, pois não 
		encontramos (ainda) na literatura tal exposição
		concisa. De qualquer forma, o conceito e o procedimento
		para obtenção das redes a partir de dados é usual,
		a exposição neste texto e no artigo~\cite{timeS} serve
		principalmente ao intuito de formalização do
		processo.
	\item A Seção~\ref{sec:sec} é dedicada ao ``Seccionamento de Erdös'', para obtenção dos três setores básicos da rede, compostos por: hubs, intermediários e periféricos. O método parece não ter sido aplicado antes para este fim, e é resultado imediato da observação das caudas longas de dados reais contrastadas com a rede Erdös-Rényi~\cite{3setores}.
\end{itemize}


\section{Estatística temporal e circular}

Para observação de padrões temporais, foram consideradas escalas diferentes.
Em cada escala, de segundos e meses, foram construídos histogramas de atividade
e feitas algumas medidas de estatística circular.
A fração $\frac{b_h}{b_l}$ entre a maior $b_h $ e a menor $b_l$ incidência nos histogramas
serviram como pista sobre quão uniforme são as distribuições observadas.

Considere cada $ medida $ (dado pontual) como um número complexo com módulo 1, $z=e^{i\theta}=\cos(\theta)+i\sin(\theta)$, onde $\theta= medida \frac{2\pi}{periodo}$. Os momentos $m_n$, tamanhos dos momentos $R_n$, ângulo médio $\theta_\mu$, e o ângulo médio reescalado $\theta_\mu'$ são definidos assim:
\begin{align}\label{eq:cmom}
	m_n&=\frac{1}{N}\sum_{i=1}^N z_i^n \nonumber\\
	R_n&=|m_n|\\
	\theta_\mu&=Arg(m_1) \nonumber \\
	\theta_\mu'&=\frac{period}{2\pi} \theta_\mu \nonumber
\end{align}

$\theta_\mu'$ é usado como medida de localização. A dispersão é medida usando a variância circular $Var(z)$, 
o desvio padrão circular $S(z)$, e a dispersão circular $\delta(z)$:

\begin{align}\label{eq:cmd}
	Var(z)&=1 - R_1 \nonumber\\
	S(z)&= \sqrt{-2\ln(R_1)}\\
	\delta(z)&=\frac{1-R_2}{2 R_1^2} \nonumber
\end{align}

Como esperado, há uma correlação positiva entre $Var(z)$, $S(z)$ e $\delta(z)$,
como pode ser notado nas informações de suporte de~\cite{timeS}.
A medida $\delta(z)$ foi preferida na discussão dos resultados.

\subsection{Formação das redes de interação}\label{sec:intNet}
Redes de interação podem ser modeladas tanto com quanto sem peso, tanto dirigida quando não dirigida~\cite{bird,newmanCommunityDirected,newmanCommunity2013,newmanBook}.
Neste trabalho, quando possível, consideramos redes dirigidas e com peso, a mais informativa das possibilidades.
Nestes casos, desconsideramos as versões dirigidas sem peso, não dirigidas com peso e não dirigidas e sem peso.

Em geral, as redes de interação são obtidas da seguinte forma:
uma reação direta do participante B a uma mensagem do participante A implica em uma aresta de A para B,
representando a informação que foi de A para B.
O raciocínio é: se B reagiu
a uma mensagem de A, ele/ela leu o que A escreveu e formulou uma reação, portanto B
assimilou informação de A, assim $A \rightarrow B$.
A inversão da direção da aresta produz a rede de status: B leu a mensagem e considerou
o que A escreveu digno de resposta, dando status para A, portanto $B \rightarrow A$.
Neste trabalho, as redes de interação são dirigida conforme o fluxo de informação, $A \rightarrow B$.
A Figura~\ref{formationNetwork} expõe esta formação. Maiores detalhes são:arestas em ambas as direções são consideradas distintas;
selfloops são consideradas não informativas (para os interesses atuais) e descartadas;  
a primeira interação $A\rightarrow B$ cria a aresta com peso um; 
a cada nova interação $A\rightarrow B$ um é adicionado ao peso da aresta.
Estas redes de interação humana constam na literatura como portadoras
de propriedades livres de escala (e pequeno mundo), como esperado
para (algumas) redes sociais~\cite{bird,newmanBook}.

\begin{figure}[!h]
	\centering
	\includegraphics[width=0.5\textwidth]{figs/criaRede__}
	\caption{A formação da rede de interação a partir
		de mensagens e respostas.
		Cada vértice representa um participante.
		Uma resposta do participante B a uma
		mensagem do participante A é considerada
		evidência de que B recebeu informação de A,
		representada então por uma aresta dirigida.
		Múltiplas mensagens adicionam ``peso'' à
		aresta dirigida. Maiores detalhes
		estão na Seção~\ref{sec:intNet}}
	\label{formationNetwork}
\end{figure}

\section{Seccionamento de Erdös}\label{sec:sec}
Em uma rede livre de escala, os setores periféricos, intermediários
e de hubs podem ser observados através de uma comparação
com uma rede de Erdös Rényi com o mesmo número de arestas e vértices~\cite{3setores}, como na Figura~\ref{fig:setores}.
Referiremos-nos a este procedimento como \emph{seccionamento de Erdös},
com os setores resultantes chamados \emph{setores de Erdös} (ou \emph{setores primitivos}, \emph{setores básicos} da rede).

A distribuição de grau
$\widetilde{P}(k)$
de uma rede livre de escala ideal
$\mathcal{N}_f$ com $N$ 
vértices e $z$ arestas possui menos
vértices com grau médios do que a distribuição $P(k)$
de uma rede Erdös-Rényi com o mesmo número de vértices e arestas.
De fato, definimos (neste trabalho) o setor intermediário de uma
rede como sendo o conjunto de todos os vértices cujo grau é
menos abundante em uma rede real do que no modelo de Erdös-Rényi:
\begin{equation}\label{criterio}
	\widetilde{P}(k)<P(k) \Rightarrow \text{k é grau intermediário}
\end{equation}

Se $\mathcal{N}_f$ for dirigida e não possuir selfloops, a probabilidade
de existência de uma aresta entre dois vértices arbitrários é
$p_e=\frac{z}{N(N-1)}$.
Um vértice em um dígrafo de Erdös-Rényi com o mesmo número de vértices e aresta,
portanto mesma probabilidade
$p_e$
para existência de aresta, terá grau $k$ com probabilidade:
\begin{equation}
	P(k)=\binom{2(N-1)}{k}p_e^k(1-p_e)^{2(N-1)-k}
\end{equation}

A cauda longa de graus baixos consiste nos vértices de borda,
i.e. o setor periférico ou periferia, onde
$\widetilde{P}(k)>P(k)$ e $k$ é mais baixo
que qualquer valor intermediário de $k$.
A cauda londa de grau alto é o setor dos hubs,
i.e.
$\widetilde{P}(k)>P(k)$ e $k$ é maior que qualquer valor de $k$ do setor
intermediário.
O raciocínio para esta classificação é: os vértices tão conectados
que são virtualmente inexistentes em redes conectadas por puro acaso
(i.e. sem ligação preferencial) são corretamente associadas
aos hubs.
Vértices com pouquíssimas conexões, e muito mais abundantes do que esperado
por puro acaso, são atribuídos à periferia.
Vértices com valores de grau previstos como os mais abundantes caso
as conexões sejam fruto de puro acaso, valores próximos da média,
e menos abundantes em nas redes reais, são classificados como intermediários.


\begin{figure}[!h]
	\centering
	\includegraphics[width=0.5\textwidth]{figs/fser_}
	\caption{As distribuições de grau de modelos ideais de redes livres de escala e Erdös-Rényi
		A segunda possui mais vérices intermediários, enquanto a primeira possui mais vértices
		periféricos e hubs. As bordas dos setores são definidas pelas duas intersecções $k_L$
		e $k_R$ das distribuições de conectividade.
		Os graus característicos estão nos intervalos compactos:
                $[0,k_L]$, $(k_L,k_R]$, $(k_R,k_{max}]$
	para os setores de Erdös (periferia, intermediários e hubs).}	
		\label{fig:setores}
\end{figure}

Para assegurar a validade estatística dos histogramas, os intervalos podem
ser escolhidos de forma que contenham ao menos $\eta$ vértices da rede real.
Assim, cada intervalo, começando no grau $k_i$, estende-se por
$\Delta_i=[k_{i},k_{j}]$,
onde $j$ é o menor inteiro tal que há ao menos $\eta$ vértices com grau maior que ou igual a $k_i$,
e menos que $k_j$. Isso altera a equação~\ref{criterio} para:

\begin{equation}\label{criterio2}
	\sum_{x=k_i}^{k_j} \widetilde{P}(x) < \sum_{x=k_i}^{k_j} P(x) \Rightarrow \text{i é intermediário}
\end{equation}

Se a força $s$ for usada para comparação, $P$ permanece a mesma, mas
$P(\kappa_i)$ com $\kappa_i=\frac{s_i}{\overline{w}}$
deve ser usado na comparação, com
$\overline{w}=2\frac{z}{\sum_is_i}$
o peso médio da aresta e $s_i$ o peso do vértice $i$.
Para graus de entrada e saída
($k^{in}$, $k^{out}$)
a comparação com a rede real deve ser feita com:
\begin{equation}
	\hat{P}(k^{way})=\binom{N-1}{k^{way}}p_e^k(1-p_e)^{N-1-k^{way}}
\end{equation}

\noindent where \emph{way} (sentido) pode ser \emph{in} or \emph{out} (entrada e saída).
Forças de entrada e saída ($s^{in}$, $s^{out}$)
são divididas por
$\overline{w}$
e comparadas também usando $\hat{P}$. Note que $p_e$ permanece a mesma,
pois cada aresta é uma aresta de entrada (ou de saída), e há no máximo $N(N-1)$
arestas entrando (ou saindo), portanto
$p_e=\frac{z}{N(N-1)}$
assim como no caso do grau total

Em outras palavras, seja $\gamma$ e $\phi$ inteiros no intervalos.
$1 \leq \gamma \leq 6$, $1 \leq \phi \leq 3$,
e cada uma das seis possibilidades de seccionamento de Erdös 
$\{E_{\gamma}\}$
possui três setores de Erdös
$E_{\gamma}= \{e_{\gamma, \phi} \}$
definidos como:

\begin{alignat}{3}\label{eq:part}
	e_{\gamma,1}&=\{\;i\;|\;\overline{k}_{\gamma,L}\geq&&\overline{k}_{\gamma,i}\} \nonumber \\
	e_{\gamma,2}&=\{\;i\;|\;\overline{k}_{\gamma,L}<\;&&\overline{k}_{\gamma,i}\leq\overline{k}_{\gamma,R}\} \\ 
	e_{\gamma,3}&=\{\;i\;|\;&&\overline{k}_{\gamma,i}<\overline{k}_{\gamma,R}\} \nonumber
\end{alignat}

\noindent onde $\{\overline{k}_{\gamma,i}\}$ é:

\begin{equation}
	\begin{split}
		\overline{k}_{1,i}&=k_i \\
		\overline{k}_{2,i}&=k_i^{in} \\
		\overline{k}_{3,i}&=k_i^{out} \\
		\overline{k}_{4,i}&=\frac{s_i}{\overline{w}} \\
		\overline{k}_{5,i}&=\frac{s_i^{in}}{\overline{w}} \\
		\overline{k}_{6,i}&=\frac{s_i^{out}}{\overline{w}} \\
	\end{split}
\end{equation}

\noindent e ambos $\overline{k}_{\gamma,L}$ e $\overline{k}_{\gamma,R}$ são encontrados  usando
$P(\overline{k})$ ou $\hat{P}(\overline{k})$
como descrito acima.

Como métricas diferentes podem ser usadas para identificas
os três tipos de vértices, critérios compostos podem ser
definidos. Após uma inspeção cuidadosa das possibilidades,
os critérios compostos foram reduzidos a 6: $\{C_\delta\}_{\delta=1}^{6}$.
Utilizando as Equações~\ref{eq:part}, estes critérios compostos $C_\delta$, com $\delta$ inteiro no intervalo $1\leq\delta<6$ podem ser descritos como:

\begin{equation}
	\begin{split}
		%\begin{multline}
		C_1&=\{c_{1,\phi}=\left\{i\mid i\;\in e_{\gamma,\phi}, \;\forall\; \gamma\}\right\} \\
		C_2&=\{c_{2,\phi}=\left\{i\mid \exists \;\;\gamma: i \in e_{\gamma,\phi}\}\right\} \\
		C_3&=\{c_{3,\phi}=\left\{i\mid i\;\in e_{\gamma,\phi'}, \;\forall\; \gamma,\;\forall\;\phi'\geq \phi\}\right\} \\
		C_4&=\{c_{4,\phi}=\left\{i\mid i\;\in e_{\gamma,\phi'}, \;\forall\; \gamma,\;\forall\;\phi'\leq \phi\}\right\} \\
		C_5&=\{c_{5,\phi}=\left\{i\mid i\;\in e_{\gamma,\phi'}, \;\forall\; \gamma,\right.\\
																																	&\;\;\;\;\;\;\;\;\;\;\;\;\;\;\;\;\;\; \left.\;\forall\;(\phi'+1)\%4\leq (\phi+1)\%4\}\right\} \\
		C_6&=\{c_{6,\phi}=\left\{i\mid i\;\in e_{\gamma,\phi'}, \;\forall\; \gamma,\right.\\
																																	&\;\;\;\;\;\;\;\;\;\;\;\;\;\;\;\;\;\; \left.\;\forall\;(\phi'+1)\%4\geq (\phi+1)\%4\}\right\} \\
		%\end{multline}
	\end{split}
\end{equation}


No artigo~\cite{timeS}, os critérios $C_1$, $C_3$ e $C_5$ foram chamados exclusivistas, os critérios $C_3$ e $C_4$ de cascata e os critérios $C_5$ e $C_6$ de externos. Note que uma cascata exclusivista $C_3$ é a mesma classificação que uma cascata invertida (considera-se dos periféricos aos hubs) e inclusivista. Estes critérios compostos são especialmente úteis para observar estruturas com poucos participantes ou fruto de pouca atividade (veja as figuras do documento de Supporting Information de~\cite{timeS}).

\section{Média e desvio do PCA ao longo do tempo}
A Análise de Componentes Principais (PCA é a sigla consagrada, do inglês Principal Component Analysis) foi usada para observar a estabilidade
na formação das componentes. Mais especificamente: 1) quais as medidas que contribuiem para cada componente e em que medida;
2) quanto da dispersão está concentrada em cada componente.

\subsection{Medidas consideradas e acrescentadas}
The topology of the networks was studied using Principal Component Analysis (PCA~\cite{pca}) with a small selection of the most basic and fundamental measurements for each vertex, as follows:

\begin{itemize}
	\item Degree     $k_i$: number of edges linked to vertex $i$.
	\item In-degree  $k_i^{in}$: number of edges ending at vertex $i$.
	\item Out-degree $k_i^{out}$: number of edges departing from vertex $i$.
	\item Strength $s$: sum of weights of all edges linked to vertex $i$.
	\item In-strength $s_i^{in}$: sum of weights of all edges ending at vertex $i$.
	\item Out-strength $s_i^{out}$: sum of weights of all edges departing from vertex $i$.
	\item Clustering coefficient $cc_i$: fraction of pairs of neighbors of $i$ that are linked.  The standard clustering coefficient for undirected graphs was used.
	\item Betweenness centrality $bt_i$: fraction of geodesics that contain vertex $i$. The betweenness centrality index considered directions and weight, as specified in~\cite{faster}.
\end{itemize}

In order to capture symmetries in the activity of participants, the following metrics were introduced for a vertex $i$: 

\begin{itemize}
	\item Asymmetry: $asy_i=\frac{k_i^{in}-k_i^{out}}{k_i}$.
	\item Mean of asymmetry of edges: $\mu_i^{asy}=\frac{\sum_{j\in J_i} e_{ji}-e_{ij}}{|J_i|=k_i}$, where $e_{xy}$ is 1 if there is an edge from $x$ to $y$, and $0$ otherwise. $J_i$ is the set of neighbors of vertex $i$, and $|J_i|=k_i$ is the number of neighbors of vertex $i$.
	\item Standard deviation of asymmetry of edges: $\sigma_i^{asy}=\sqrt{\frac{\sum_{j\in J_i}[\mu_{asy} -(e_{ji}-e_{ij}) ]^2  }{k_i}  }$.
	\item Disequilibrium: $dis_i=\frac{s_i^{in}-s_i^{out}}{s_i}$.
	\item Mean of disequilibrium of edges: $\mu_i^{dis}=\frac{\sum_{j \in J_i}\frac{w_{ji}-w_{ij}}{s_i}}{k_i}$, where $w_{xy}$ is the weight of edge $x\rightarrow y$ and zero if there is no such edge.
	\item Standard deviation of disequilibrium of edges: $\sigma_i^{dis}=\sqrt{\frac{\sum_{j\in J_i}[\mu_{dis}-\frac{(w_{ji}-w_{ij})}{s_i}]^2}{k_i}}$.
\end{itemize}

\section{Teste de Kolmogorov-Smirnoff para os textos produzidos por cada setor}

Be $F_{1,n}$ and $F_{2,n'}$ two empirical distribution functions, where $n$ and $n'$ are the number of observations on each sample. The two-sample Kolmogorov-Smirnov test rejects the null hypothesis if:
\begin{equation}\label{eq:ks}
D_{n,n'} > c(\alpha)\sqrt{\frac{n+n'}{nn'}}
\end{equation}

where $D_{n,n'}=sup_x[F_{1,n}-F_{2,n'}]$ and $c(\alpha)$ is given for each level of $\alpha$:

\begin{table}[H]
\centering
\begin{tabular}{|l||c|c|c|c|c|c|}\hline
$\alpha$    & 0.1  & 0.05 & 0.025 & 0.01 & 0.005 & 0.001 \\\hline
$c(\alpha)$ & 1.22 & 1.36 & 1.48  & 1.63 & 1.73  & 1.95  \\\hline
\end{tabular}
\end{table}

\subsection{Adaptação}
We need to compare empirical distribution functions, so $D_{n,n'}$ is given, as are $n$ and $n'$. Therefore, as all terms in equation~\ref{eq:ks} are positive and $c(\alpha)$ can be isolated:

\begin{equation}\label{eq:ks2}
c(\alpha) < \frac{D_{n,n'}}{\sqrt{\frac{n+n'}{nn'}}} = c'(\alpha)
\end{equation}

Tables~\ref{tab:kolSub}-\ref{tab:kolPctInter} are populated with values for $c'(\alpha)$. When $c'(\alpha)$ is high, low values of $\alpha$ are possible for the test to reject the null hypothesis. Therefore, when $c'(\alpha)$ is greater than $\approx 1.7$, it is reasonable to assume that $F_{1,n}$ and $F_{2,n'}$ differ.


%\section{Audiovisualization of data}
\section{Audiovisualização de dados}

Networks were visualized with animations, image galleries and online gadgets developed specifically for this research~\cite{animacoes,galGMANE,appGMANE}. Such visualizations were crucial to guide research into the most important features of network evolution. Furthermore, the size of the three Erd\"os sectors could be visualized in a timeline fashion. Visualization of network structure was especially useful in the inspection of data and derived structures from the email lists.

%\section{Typological considerations}
\section{Considerações tipológicas}
Por enquanto, isso para nós quer dizer:
\begin{itemize}
	\item Consideração sobre o potencial estigmatizante da classificação.
	\item Considerações sobre o indivíduo ou a rede que é classificada, i.e. recebe um tipo.
\end{itemize}

\section{Web semântica}
Para a formalização de conceitualizações, e
para formatos de dados apropriados para armazenamento,
compartilhamento e referência,
foram adotadas as recomendações de linked data
da W3C~\cite{rdf,owl,rdfs}.

\subsection{A construção de ontologias OWL}
Foram construídas ontologias e vocabulários a partir de entrevistas (OBS e VBS principalmente). Um método de levantamento de ontologia orientado aos dados surgiu, potencialmente útil a todos os portais e software em necessidade de ontologias, e foi responsável por 2 ontologias (OPA dados e OCD).

\subsection{A triplificação de dados relacionais}
Para disponibilização e uso de dados de diferentes fontes, foram
feitos pequenos programas de computador (scripts) para
acessar dados relacionais e escrever triplas RDF.

\chapter{Resultados}
%\section{Time stability in human interaction networks}
\section{Estabilidade temporal em redes de interação humana}
Explicitados em~\cite{timeS}.
\section{Criação da nuvem brasileira de dados participativos}
%\subsection{RDF data conversion of data into linked data}
Primeiras ontologias, aplicações, dados, publicações, etc.
Produtos 3,4,5 e artigo sobre OPS.
\subsection{Conversão de dados relacionais para RDF}
Roteiros de triplificação do Participa.br, Cidade Democrática, AA.
%\subsection{Published linked data and OWL ontologies}
\subsection{Síntese de ontologias OWL}
Scripts para síntese, seja livre ou a partir dos dados.
\subsection{Ontologias e dados publicados}
datahub.io, MP, arXiv.

\section{Aproveitamento}
\subsection{Procedimentos de percolação social}
%\subsection{Recommendation systems for the enrichment of semantic navigation}
Procedimentos cíclicos e procedimentos efèmeros.
Experimentos paradigmáticos:
\begin{itemize}
	\item Crowdfunding para a pesquisa de redes (Vaca do fim do mundo).
	\item Escrita para os de maior betweenness e de maior closeness, mostrar que não bate para as listas.
\end{itemize}

\subsection{Sistemas de recomendação para o enriquecimento da navegação semântica de recursos}
%\subsection{Understanding the social being}
Como descrito no produto 4 PNUD. Repassar caracterização básica e direcionar ao escrito.

\subsection{Compreensão sobre a entidade social}
Scale free as the consequence of $T^2$ signal, fractal, constant, with three primitive parts and greater specialization. Gradus ad Parnassum.

%\chapter{Finished and planed tasks, chronogram}
\chapter{Afazeres e cronograma}
% Colocar tabela com os comparativos
% Referenciar para complementos no texto.
\section{Experimentos}\label{sec:exp}
\section{Documents}
\subsection{To be finished}
\subsubsection{Anthropological physics}
The study of human systems raises conceptual
and ethical issues that require anthropological considerations.
There are two immediate routes to this concepts:
\begin{itemize}
	\item What data should or can be used?
	\item Can one experiment in a network of humans? In which context?
\end{itemize}

The short answer is that ethics committees and procedures are dedicated to dealing with those issues.
Even so, there is a key-concept from the anthropological legacy: the study of the self as exposed to the interested culture or context. 
In this sense, it is reasonable (if not a suggestion) that
a researcher do reflexive consideration, 
i.e. that he/she observe and make assumptions about its own sampling of the world.
Within this same framework, many social networks (email, Facebook, Twitter, Participa.br, AA) were openly mined,
with feedback to and from the studied communities.
The term ``anthropological physics'' started being used in Brazil around
2014 and can be thought as a subfield of Social Physics.

\subsubsection{Gradus}\label{sec:gradus}
Uma lista detalhada de ambiguidades e sinônimos deverá completar o que está na Seção~\ref{sec:jar}.

% 1/N^2 == physics of humans, they are the agents
Fazer o $2^(100 2)$

Consider a idealized constitution of these networks: 
\begin{itemize}
	\item the resources of the environment are the persons, each with an amount of time available.
	\item The amount of resource employed by the environment to the network is constant through all connective sectors
\end{itemize}

\subsection{Finished}


\section{Cronograma}\label{sec:cron}


\begin{table}[h!]
\label{tab:cron}
\begin{center}
  \begin{tabular}{ | c ||   c | c |     c | c |   c | c |}
    \hline
      & \multicolumn{2}{|c|}{2013} & \multicolumn{2}{|c|}{2014} & \multicolumn{2}{|c|}{2015} \\
    \hline
    Atividade & 1$^{\circ}$ & 2$^{\circ}$ & 1$^{\circ}$ & 2$^{\circ}$ & 1$^{\circ}$ & 2$^{\circ}$ \\
    \hline \hline

    1 & [$\bullet$] & [$\bullet$] & $\bullet$ & $\bullet$ & & \\
    \hline
    2 & [$\bullet$] & [$\bullet$] & [$\bullet$] & [ ] & [ ] & [ ] \\
    \hline
    3 & [ ] & [$\bullet$] & [$\bullet$] & [$\bullet$] & [$\bullet$] & [ ]  \\
    \hline
    4 & [ ] & [$\bullet$] & [$\bullet$] & [$\bullet$] & [$\bullet$] & [$\bullet$]  \\
    \hline
    5 & & & & & [$\bullet$] & [$\bullet$]  \\
    \hline
    6 & [$\bullet$] & [$\bullet$] & [$\bullet$] & [$\bullet$] & [$\bullet$] & [$\bullet$]  \\
    \hline
    7 & [$\bullet$] & [$\bullet$] & [$\bullet$] & [$\bullet$] & [$\bullet$] & [$\bullet$]  \\
    \hline
  \end{tabular}
  \caption{Cronograma de atividades ao longo dos semestres, descritas na Seção~\ref{sec:cron}.
	  A marcação $\bullet$ indica previsão feita no início do doutorado.
  A marcação [ ] se refere ao relato e previsão, agora no final do $1^{\circ}$ semestre de 2015.
  As principais diferenças do previsto foram: as disciplinas foram terminadas no primeiro ano; a revisão da literatura, os acréscimos aos modelos atuais com o foco no participante da rede, e a implementação computacional, estas três atividades estão sendo realizadas constantemente e devem durara até pouco antes da entrega e defesa da tese. }
\label{tab:cron}
\end{center}
\end{table}


Este projeto foi inicialmente dividido segundo as etapas a seguir 
e usadas como referência na Tabela~\ref{tab:cron}:

\begin{enumerate}
\item Créditos Obrigatórios: cumprimento dos créditos obrigatórios em disciplinas, exigidas pelo programa de Doutorado do IFSC/USP.

\item Revisão da literatura.

\item Acréscimos aos modelos atuais com o foco no participante da rede.

\item Implementação computacional.

\item Escrita da tese.

\item Escrita e publicação dos resultados em artigos.

\item Trocas com pessoas externas, estabelecimento de colaborações.
\end{enumerate}

Considerações sobre estes itens:
\begin{enumerate}
\item Foram cursadas as disciplinas de Processamento de linguagem natural, Mineração de dados, Visualização de dados e Web semântica. Dediquei um ano inteiramente às disciplinas. Estranhamente, fechei todas com B. No mestrado, fazia mais de 30 créditos na graduação, 4 disciplinas na pós, pesquisa, e fechei todas com A.
\item A literatura de para o trabalho proposto é ampla e este aprofundamento tem sido constante,
\item Os acréscimos aos modelos atuais tem tido o foco no participante da rede.
\item Há implementação computacional de provas de conceito, bibliotecas, rotinas básicas e rotinas para replicar resultados do grupo de pesquisa.

\item A escrita da tese pode tomar vários rumos: pode consistir de um conjunto de artigos ou de uma monografia final. Acho mais provável que seja um conjunto de artigos centrados no descrido na Seção~\ref{sec:gradus}.
\item Conseguimos finalizar um artigo~\cite{timeS}. Há ao menos mais um em condições de publicação~\cite{ops} e outro mais indiretamente relacionado sobre música~\cite{mass}. Além destes, há mais estes artigos no arXiv~\cite{ensaio,connectiveDiff,XXX,YY}, todos referentes ao trabalho do doutorado. Foram publicados em revista internacional os artigos AA e Images/Vilson, ambos sem a colaboração do orientador.
\item Parte substancial do trabalho consistiu em experimentos de coleta e difusão de informação, o que disparou reuniões, visitas e colaborações. Este processo foi iniciado logo antes do doutorado e pode ser apreciado, por exemplo, pelas visitas a São Carlos de parcerios de pesquisa, pela integração do pesquisador ao grupo de pesquisa Nexus, vinculado ao CNPq, e ao aporte do PNUD/ONU dado ao pesquisador, sobre o qual a Presidência da República se posicionou como beneficiária.
\end{enumerate}

\chapter{Conclusões}
Há, a princípio, uma confirmação de que os conhecimentos de redes complexas
possuem aplicações diversas e potencialmente benéficas para o participante.
Por exemplo, os experimentos apresentaram modificações da estrutura social para comportar a pesquisa, e podem ser usados para comportar outros empreendimentos. Os estudos de estabilidade e diferenciação em redes de interação humana apontam na direção de tipologias de redes e de participantes, com base nos setores de Erdös.
Um legado de dados ligados e abertos é conveniente para apresentar estes dados às comunidades acadêmicas e interessadas nas aplicações, para o qual foram adiantadas ontologias, vocabulários, rotinas de conversão de dados relacionais em RDF e os dados em si.

Uma direção simples
é focar no \emph{Complex Networks Gradus ad Parnassum},
que é um manual para introdução à àrea através do benefício 
das redes complexas para o indivíduo pela ação dele próprio.
Uma via menos pedagógica, mas mais fundamental, é
explorar as estabilidades encontradas: até que número de
agentes a distribuição dos setores e a formação do PCA se mantém?;
como caracterizar a intermitência dos agentes enquanto a distribuição
grau é estável?

Há, entretanto, considerações teóricas profundas sobre a área,
com implicações sobre a própria constituição das redes complexas.
Ao mesmo tempo, os métodos utilizados também já parecem novos,
e está faltando uma capacidade de apresentar isso para as revistas.
Talvez por falta de equipe adequada para realizar o trabalho,
talvez por falta de prática do doutorando na escrita e encaminhamento
acadêmico.



\bibliography{biblio}

\end{document}

