\documentclass[a4paper,openright,12pt]{report} %,parskip=full

\usepackage{amssymb,amsmath,textcomp}
\usepackage[brazil]{babel}
\usepackage[utf8]{inputenc}
\usepackage{braket}
\usepackage{booktabs}  %Possibilidade de linhas mais grossas nas tabelas.

%MEU
%\usepackage[brazilian, portuguese, activeacute]{babel}
%\usepackage{graphicx}
\usepackage{wrapfig}
\usepackage{subfigure}
%\usepackage{color}
%\usepackage{amssymb}
\usepackage{amsmath}  %for binom, not
%\usepackage{pifont}   %for ding
\usepackage{hyperref}
%\usepackage{amsthm}   %for what
%\usepackage{helvet}   %for what
\usepackage{cancel}   %for cancel
\hypersetup{
    colorlinks,
    citecolor=black,
    filecolor=black,
    linkcolor=black,
    urlcolor=black
}

\usepackage{etex}

\usepackage{cite}



%\usepackage[footnotesize,hang]{caption} 

%\usepackage[hang,small,labelsep=endash]{caption} % Hifen na legenda de tableas e figuras
\usepackage[hang,footnotesize,labelsep=endash,tableposition=top]{caption} % Hifen na legenda de tableas e figuras
%CHCHCH
%\usepackage{subcaption}

%%%%%%%%%%%%%%%%% Bibliografia
\usepackage{url}
\usepackage[bibjustif,abnt-thesis-year=both,num]{abntcite} %%%%%% ABNY
%%%%%%%%%%%%%%%%% Espacamento
%\makeatletter  %Espa�amento entre os itens da bibliografia
%\newcommand{\adjustmybblparameters}{\setlength{\itemsep}{2\baselineskip}\setlength{\parsep}{0.5ex}}
%\let\ORIGINALlatex@openbib@code=\@openbib@code  
%\renewcommand{\@openbib@code}{\ORIGINALlatex@openbib@code\adjustmybblparameters}
%\makeatother
%%%%%%%%%%%%%%%%%%%%%%%%%%%%%%%%%%%
% Espacaçamento entre as referências
\makeatletter  %Espa�amento entre os itens da bibliografia
\newcommand{\adjustmybblparameters}{\setlength{\itemsep}{2\baselineskip}\setlength{\parsep}{0.5ex}}
\let\ORIGINALlatex@openbib@code=\@openbib@code  
\renewcommand{\@openbib@code}{\ORIGINALlatex@openbib@code\adjustmybblparameters}
\makeatother

%%%%%%%%%%%%%%%%%%%%%%%%%%%%%%%%%%%%%%%%%%%%%%%
%%%%%%%%%%%%%%%%%%%%%%%%%%%Symbol footnote
\long\def\symbolfootnote[#1]#2{\begingroup%
\def\thefootnote{\fnsymbol{footnote}}\footnote[#1]{#2}\endgroup}
%%%%%%%%%%%%%%%%%%%%%%%%%%

\usepackage[Conny]{fncychap}
\usepackage{appendix}

%%%%%%%%%%%%%%%%%%%%%%%%%%%%%%%%%%%%%%%5

\makeatletter
\renewcommand*{\@makechapterhead}[1]{%
  \vspace*{-1.0cm}%
  {\parindent \z@ \raggedright \normalfont
    \ifnum \c@secnumdepth >\m@ne
      \if@mainmatter%%%%% Fix for frontmatter, mainmatter, and backmatter 040920
        \DOCH
      \fi
    \fi
    \interlinepenalty\@M
    \if@mainmatter%%%%% Fix for frontmatter, mainmatter, and backmatter 060424
      \DOTI{#1}%
    \else%
      \DOTIS{#1}%
    \fi
  }}

%older
%%%%%%%% Espaçamento entre títulos de seções e linhas anteriores e subsequentes
%\usepackage[compact]{titlesec}
%\titlespacing{\section}{0pt}{0.5cm}{0.5cm}
%%\titlespacing{\chapter}{0pt}{2.0cm}{2.0cm}
%\titlespacing{\subsection}{0pt}{0.5cm}{0.5cm}
%\titlespacing{\subsubsection}{0pt}{0.5cm}{0.5cm}	

%%abnt mestrado
%%%%%%%%% Espaçamento entre títulos de seções e linhas anteriores e subsequentes
%\usepackage{titlesec} %[compact]
%\titlespacing{\chapter}{0pt}{1.9cm}{2.2cm}
%\titlespacing{\section}{0pt}{1.9cm}{2.2cm}
%\titlespacing{\subsection}{0pt}{2.1cm}{2.2cm}
%\titlespacing{\subsubsection}{0pt}{2.2cm}{2.2cm}	

%quali
%%%%%%%% Espaçamento entre títulos de seções e linhas anteriores e subsequentes
\usepackage{titlesec} %[compact]
\titlespacing{\chapter}{0pt}{0cm}{1.5cm}
\titlespacing{\section}{0pt}{1.5cm}{1.5cm}
\titlespacing{\subsection}{0pt}{1.5cm}{1.5cm}
\titlespacing{\subsubsection}{0pt}{1.5cm}{1.5cm}	

%  \usepackage{titlesec}
  \titleformat{\chapter}[display]
  {\normalfont\bfseries\filcenter}
  {\LARGE\thechapter}
  {-2ex}
  {%\titlerule[2pt]
	 % \vspace{2ex}%
  \LARGE}
  [%\vspace{1ex}%
  {\titlerule[1pt]}]


%%%%%%%%%% Escreve ``Figura`` e ``Tabela`` nas listas de figuras e tabelas
%%CHCHCH
%\usepackage[titles]{tocloft}
%\renewcommand{\cftfigaftersnum}{ -}
%\renewcommand{\cfttabaftersnum}{ -}
%\renewcommand{\cftfigpresnum}{Figura }
%\renewcommand{\cfttabpresnum}{Tabela }
%\setlength{\cftfignumwidth}{2.2cm}
%\setlength{\cfttabnumwidth}{2.2cm}
%%%%%%%%%%%%%%%%%%%%%%%%%%%%%%%%


\usepackage[pdftex]{graphicx}  %pacote para colocar figuras jpg
\usepackage[final]{pdfpages}  %include pfd coloca um pdf em uma pagina

%\usepackage[T1]{fontenc} %CHCHCH
%\usepackage[OT2,T1]{fontenc} %CHCHCH
%\usepackage[T1,T2A]{fontenc} %needs t2aenc.def
\usepackage[T1,OT2,OT1]{fontenc} %CHCHCH

%\usepackage{cyrillic} 

%russian
\newcommand\cyr{%
\renewcommand\rmdefault{wncyr}%
\renewcommand\sfdefault{wncyss}%
\renewcommand\encodingdefault{OT2}%
\normalfont
\selectfont}
\DeclareTextFontCommand{\textcyr}{\cyr}

%%CHCHCH
%%russian
%\newcommand{\cyrrm}{\fontencoding{OT2}\selectfont\textcyrup}
%\newcommand{\cyrit}{\fontencoding{OT2}\selectfont\textcyrit}
%\newcommand{\cyrsl}{\fontencoding{OT2}\selectfont\textcyrsl}
%\newcommand{\cyrsf}{\fontencoding{OT2}\selectfont\textcyrsf}
%\newcommand{\cyrbf}{\fontencoding{OT2}\selectfont\textcyrbf}
%\newcommand{\cyrsc}{\fontencoding{OT2}\selectfont\textcyrsc}
%%%%% cyrrm = "Roman", or really upright, normal font
%%%%% cyrit = Italic (cursive forms of letters)
%%%%% cyrsl = Italic (non-cursive forms of letters)
%%%%% cyrsf = Sans-serif
%%%%% cyrbf = Bold-face 




%\usepackage[math]{iwona}
\renewcommand{\familydefault}{\sfdefault}
%\renewcommand{\familydefault}{\sfdefault} math

\usepackage[absolute]{textpos}
\usepackage[retainorgcmds]{IEEEtrantools}
\usepackage{leftidx}

%CHCHCH
%\usepackage[superscript]{cite}  %citacoes em superscript
\usepackage{cite}  %nao, meu, citacoes em superscript nao!

\usepackage{tikz}
\usetikzlibrary{arrows}
\usetikzlibrary{decorations}
\usetikzlibrary{snakes}

\usepackage{color}

\definecolor{DarkBlue}{rgb}{0.1,0.1,0.5}
\definecolor{Red}{rgb}{0.9,0.0,0.1}
\definecolor{DarkGreen}{rgb}{0.10,0.50,0.10}

% see documentation for a0poster class for the size options here
\let\Textsize\large
\def\Head#1{\noindent\hbox to \hsize{\hfil{\LARGE\color{DarkBlue} #1}}\bigskip}
\def\LHead#1{\noindent{\LARGE\color{DarkBlue} #1}\smallskip}
\def\Aut#1{\noindent{\Huge\color{DarkBlue} #1}\smallskip}
\def\End#1{\noindent{\large\it\color{DarkBlue} #1}\smallskip}
\def\Subhead#1{\noindent{\Large\color{DarkBlue} #1}}
\newcommand{\quiteHuge}{\fontsize{120}{93}\selectfont}
\def\Title#1{\begin{center}\noindent{\quiteHuge\color{DarkGreen}#1}\end{center}}


\usepackage[twoside,inner=3cm,outer=2cm,top=3cm,bottom=2cm]{geometry}

\usepackage{setspace}   %Espacamento entre as linhas

\setstretch{1.5} %Normal da ABNT: espaçamento entre as linhas de 1.5 de linha
% Pode ser \singlespace 
%          \onehalfspace
%          \doublespace

%---------------------------------------------------------------------------------------------------------
\usepackage{fancyhdr}
\pagestyle{fancy} % colocar Capítulos, Seções, etc em minúsculo

\renewcommand{\sectionmark}[1]{\markright{\thesection\ #1}}

\fancyhf{} % deletar configuração atual do cabeçalho (header) e rodapé (foot)

\pagestyle{fancy} 

\fancyhead[LE,RO]{\thepage}

\fancyhead[LO]{\rightmark}
  
\fancyhead[RE]{\leftmark}

\renewcommand{\headrulewidth}{0.5pt}
  
\renewcommand{\footrulewidth}{0pt}

%CHCHCH
%\addtolength{\headheight}{0.5pt} % cria um espaço para linha
\addtolength{\headheight}{4.0pt} % cria um espaço para linha

\fancypagestyle{plain}{
    \fancyhead{} % exibir cabeçalho e rodapé
    \renewcommand{\headrulewidth}{0pt} % linha
}




%---------------------------------------------------------------------------------------------------------

\usepackage[marginal,symbol]{footmisc}
\footnotemargin2pt

%%%%%%%%%%%%%%%%%%


\pagestyle{empty}

%\AtBeginDocument{\addtocontents{toc}{\protect\thispagestyle{empty}}} %CHCHCH %to make tableofcontents empty pagestyle


%CHCHCH TO MAKE TABLEOFCONTENTS PAGESTULE EMPTY
%\fancypagestyle{plain}{%
%  \fancyhf{}                          % clear all header and footer fields
%  \renewcommand{\headrulewidth}{0pt}
%  \renewcommand{\footrulewidth}{0pt}
%}

\usepackage[subfigure,titles]{tocloft}
\renewcommand{\cftfigaftersnum}{ -}
\renewcommand{\cfttabaftersnum}{ -}
\renewcommand{\cftfigpresnum}{Figura }
\setlength{\cftfignumwidth}{2.2cm}%{5em}



%\renewcommand*{\chapterheadendvskip}{%
%  \vspace{3.0cm}%
%%  \vspace{0.725\baselineskip plus 0.115\baselineskip minus 0.192\baselineskip}%
%}

\usepackage{blindtext}

%%%%%%%%%%%%%%%%%%%%%%%%%%%%%%%%%%%%%%%%%%%%%%%%%%%%%%%%%%%%%%%%%%%%%%%%%%%%%%%%%%%%%%%%%%%%%%%%%%%%%%%%%%%%%%%%%%%%%%%%%%%%%%%%%%%%%%%%%%%%%%%%%%%
%%%%%%%%%%%%%%%%%%%%%%%%%%%%%%%%%%%%%%%%%%%%%%%%%%%%%%%%%%%%%%%%%%%%%%%%%%%%%%%%%%%%%%%%%%%%%%%%%%%%%%%%%%%%%%%%%%%%%%%%%%%%%%%%%%%%%%%%%%%%%%%%%%%

\begin{document}

%\setlength{\parskip}{0.5cm}%1.5ex} %0pt} % 1ex plus 0.5ex minus 0.2ex}

\hyphenation{ca-ra-te-ri-sti-cas}

%\selectlanguage{portuguese}
%\hyphenation{e-mer-g\^{e}n-cia}


\pagestyle{fancy}


\thispagestyle{empty}

\vspace{0.5cm}

\begin{center} 
%\LARGE{UNIVERSITY OF SÃO PAULO}  \\
%\LARGE{SÃO CARLOS PHYSICS INSTITUTE}
\LARGE{UNIVERSIDADE DE SÃO PAULO}  \\
\LARGE{INSTITUTO DE FÍSICA DE SÃO CARLOS}
\end{center}

\vspace{6.0cm}

\centerline{\LARGE{RENATO FABBRI}}

\vspace{3.0cm}


%\centerline{\Huge{Complex networks for the participant}} 
\centerline{\Huge{Redes complexas para o participante}} 
%\centerline{\Huge{Redes complexas para o participante de redes sociais}} 
\vspace{0.5cm}
%\centerline{\Huge{aplicada a reconhecimento de autoria}}


\vspace{6.5cm}

\begin{center}
\Large{S\~ao Carlos}\\
\Large{2015}
\end{center}



\newpage\ \thispagestyle{empty}  \newpage\thispagestyle{empty}

\setcounter{page}{1} % ABNT: deve-se começar a numeração na folha de rosto.

\begin{center}
\LARGE{RENATO FABBRI}
\end{center}

\addvspace{4.0cm}

\begin{center}
%\Huge{Complex networks for the participant}
\Huge{Redes complexas para o participante}
\end{center}

\addvspace{3.0cm}

\makebox[15cm][r]{
\begin{minipage}[l]{8cm}

\begin{singlespace}
Monografia apresentada ao Programa de Pós-Graduação em Física 
do Instituto de Física de São Carlos da Universidade de São Paulo, para o Exame de Qualifica\c c\~ao 
como parte dos requisitos para obten\c c\~ao do t\'itulo de Doutor em Ci\^encias.\\

Área de concentração: Física Aplicada

Opção: Física Computacional

Orientador: Prof. Dr. Osvaldo Novais de Oliveira Jr.
%Monograph presented to the Physics Graduate Program of the
%São Carlos Institute of Physics, University of São Paulo,
%for the qualification exam, as part of the requirements
%for obtaining the title of Doctor in Sciences.\\
%
%Concentration area: Applied Physics
%
%Option: Computational Physics
%
%Advisor: Prof. Dr. Osvaldo Novais de Oliveira Jr.
\end{singlespace}


\end{minipage}}

\addvspace{3.0cm}

\begin{center}
%\Large{Versão original} \\
\vspace{1.5cm}
\Large{S\~ao Carlos}\\
\Large{2015}
\end{center}


%%%%%%
%\thispagestyle{empty}
%
%
%\includepdf{index}
%\includepdf{fichacatalog135}
%%%%%%

%% \vspace*{8cm}
%%\begin{center}
%% \huge{FICHA CATALOGRÁFICA}

%% \vspace*{3cm}

%%\large{Elaborar ao final, quando o número de páginas da 
%%dissertação estiver definido - 
%%www.biblioteca.ifsc.usp.br/ficha}
%%\end{center}

%%%%%%%%%%%
%\newpage\thispagestyle{empty} 
%
%\vspace*{8cm}
%
%\begin{center}
%\normalsize{FOLHA DE APROVAÇÃO}
%\end{center}
%
%
%
%\newpage\ \thispagestyle{empty}  \newpage\thispagestyle{empty}\ 
%
%\vspace*{18cm}
%
%\begin{spacing}{1.2}
%
%\begin{flushright}\textit{
%\`A MINHA AV\'O,\\
%\vspace{0.5cm}
%CLELIA C\'ESPEDES ACERO
%\vspace{1cm}
%}\end{flushright}
%
%\end{spacing}
%%%%%%%%%%%



%%%%%%%%%%%%%%%%%%%%
%\newpage\ \thispagestyle{empty} %NEWW
%
%\newpage\ \thispagestyle{empty}
%
%
%\vspace*{18cm}
%
%\begin{spacing}{1.2}
%
%\makebox[15cm][r]{ 
%\begin{minipage}[l]{10cm}
%\hspace*{2.2cm}\emph{ \large{ Este trabalho foi financiado pelo \\ \hspace*{7.5cm} CNPq } }
%\end{minipage}}
%
%
%
%\end{spacing}
%%%%%%%%%%%%%%%%%%%%





%\newpage\ \thispagestyle{empty}  \newpage\thispagestyle{empty}

\newpage\ \thispagestyle{empty}  \newpage\thispagestyle{empty}


\begin{singlespace}

%\centerline{\LARGE{{\bf ABSTRACT}}}
\centerline{\LARGE{{\bf RESUMO}}}

\vspace*{1.5cm}

%\hspace*{-0.9cm} FABBRI, C. \textit{Complex networks for the participant}. 
\hspace*{-0.9cm} FABBRI, C. \textit{Redes complexas para o participante}. 
Instituto de Física de São Carlos, Universidade de São Paulo, São Carlos, 2015. 
%São Carlos Institute of Physics. University of São Paulo, São Carlos, 2015. 

\vspace*{1.2cm}

\hspace*{-0.9cm}

{\noindent
	As redes complexas compõem uma das áreas mais ativas da física recente.
	Há esforços consideráveis para apresentar estes avanços ao público não especialista, mas tudo indica
	que poucos ou nenhum são propostos para instrumentalizar o indivíduo que constitui estes sistemas a se beneficiarem.
	Ou seja, com um núcleo de conhecimento da área, e receitas para aproveitamento,
	fornecer meios para o participante interagir e entender as redes nas quais ele se encontra.
	Este trabalho busca realizar tal tarefa por meio das redes sociais do participante.
	Verificamos que tais redes exibem uma estabilidade temporal de medidas topológocas e dos
	tamanhos relativos dos setores básicos (hubs, intermediários, periféricos).
	Observamos uma diferenciação da produção de texto de cada setor básico.
	Também formalizamos as conceitualizações vinculadas a estas redes como ontologias OWL onde foi possível,
	principalmente as instâncias de participação social previstas por lei e praticadas ou implementadas como computacionalmente.
	Software e dados foram disponibilizados e usados. Protocolos escolhidos para
	facilitar a integração de estruturas de diferentes procedências,
	para reutilização dos dados em outros trabalhos e pesquisas, e para o benefício público.
	Consequências conceituais requerem considerações antropológocas e estão sendo redigidas.
	Próximos passos são: melhor documentação e desenvolvimento do aparato em software, ontologias e dados;
	considerações tipológicas das propriedades físicas observadas nas redes de interação humana, com
	atenção aos outliers, às relações entre topologia do agente e texto produzido, e à ponte com
	a bagagem mais tradicional das ciências humanas no assunto.
}

%{\noindent
%	Complex networks form one of the most active fields of recent physics.
%	With respectable efforts for exhibiting advances to the general audience, 
%	it seems, however, that few or none of these are targeted to the benefit of the individual that constitute such systems.
%	That is, with a core knowledge about the field, and recipes for harnessing,
%	provide means for the participant to interact and understand the networks he/she is in.
%	This work aims to accomplish such task by means of the social networks of the participants.
%	We verified that such networks exhibit time stability of topological measures and
%	of basic connective sector sizes and exhibit differentiation of the textual production in each basic connective sector.
%	We also formalized conceptualizations of these networks as OWL were they were possible, specially
%	in relation to the social participation instances provided by law. Finally, software and
%	data have been put available and used, as means to enable integrated analysis of different provenance and public benefit.
%	Conceptual consequences have been documented and requires anthropological considerations.
%	Furthermore, software, ontological and data contributions can be better documented and developed while
%	a typological consideration of the physical properties observed in human interaction networks should
%	bridge complex networks and the more traditional legacy of human sciences on the subject.
%}

\vspace*{1.5cm}
%\hspace*{-0.9cm} {\bf Keywords:} Complex networks. Social networks. Complexity. Anthropological physics. Linked data. Semantic web. Social participation. Text mining. Natural language processing.
\hspace*{-0.9cm} {\bf Palavras-chave:} Redes complexas. Redes sociais. Complexidade. Física antropológica.
Dados ligados. Web semântica. Participação social. Mineração de texto. Processamento de linguagem natural.

\end{singlespace}

  

%%%%%%%%%%%%%%%%%%%%%%%%%%%%%
%\newpage\thispagestyle{empty}
%
%\pagestyle{empty}
%
%\begin{singlespace}
%\listoffigures
%%\listoftables
%\end{singlespace}
%%%%%%%%%%%%%%%%%%%%%%%%%%%%%

%%%
%\newpage\ \thispagestyle{empty}  \newpage\thispagestyle{empty}

\newpage\ \thispagestyle{empty}  \newpage\thispagestyle{empty}
%\thispagestyle{empty} %CHCHCH

\tableofcontents\thispagestyle{empty}\thispagestyle{empty}%\thispagestyle{empty}

\clearpage \thispagestyle{empty}

\pagestyle{fancy}

%\chapter{Introduction}
\chapter{Introdução}
Estudos sobre redes de interação humana foram iniciados bem antes dos computadores modernos,
datam do século XIX, enquanto a fundação da ``Análise de Redes Sociais''/ARS (ou \emph{Social Network Analysis}/SNA)
é geralmente atribuida ao psiquiatra Jacob Moreno na metade do século vinte~\cite{newmanBook}.
Com a crescente disponibilidade de dados relacionados à interação humana, a pesquisa destas redes tem aumentado continuamente.
Contribuições podem ser encontradas em uma variedade de áreas, de ciências sociais e humanidades~\cite{latour2013}
a ciências sociais~\cite{bird} e física~\cite{barabasiHumanDyn,newmanFriendship},
dada a natureza multidisciplinar do assunto.
Uma das abordagens da perspectiva de uma ciência exata é representar a rede de interação como uma rede
complexa~\cite{barabasiHumanDyn,newmanFriendship},
com a qual algumas características foram reveladas.
Por exemplo, a topologia das redes de interação humana exibem um traço livre de escala,
o que aponta para a existência de um pequeno número de hubs super conectados
e um grande número de vértices pouco conectados.

Há um hiato de conhecimento e tecnologia entre o legado de redes complexas e o usufruto do participante.
Este hiato é reativo, e há evidência de que conseguirá se manter como um ecossistema de conhecimento, tecnologia
e empreendimento da sociedade em todas as suas escalas.
Deve facilitar, por exemplo:
elaboração e preparação de documentos, aquisição rápida de conhecimento,
realização de empreitadas coletivas.
Em geral: processos de coleta e difusão de informação (e bens)~\cite{pentland1,pentland2}.
% Argumentar que é tido que o correto entendimento destas estruturas
% será mais forte que XXX (ver no curso da Melanie/CE).

Este trabalho apresenta uma confirmação deste cenário e avanços.
Algumas estratégias foram selecionadas para verificar a aplicabilidade
de conceitos de redes complexas para o benefício do participane.
Em especial, experimentos muito simples parecem capazes de modificar
estruturas sociais. Neste contexto, verificamos estabilidades temporais
nas redes de interação humana, e expomos que os setores básicos
das redes (hubs, intermediários e periféricos)
produzem textos bastante diferentes entre si.
Este conhecimento é útil para uma tipologia não estigmatizante
de participantes em redes de interação.
A audiovisualização e interconexão de dados com arte e engenhocas
em software deram suporte contínuo à pesquisa científica.
Aplições foram complementadas em parceria com a Presidência da República
e o Programa das Nações Unidas para o Desenvolvimento.

A próxima seção apresenta considerações gerais sobre a literatura.
A Seção~\ref{sec:jar} expõe a proliferação de ambiguidades e sinônimos
no jargão da área.
A Seção~\ref{sec:mat} é dedicada aos dados analizados.
A Seção~\ref{sec:met} contém os métodos usados para atingir os resultados, que são explicitados na Seção~\ref{sec:res}. O cronograma de atividades e uma comparação entre afazeres planejados e finalizados
estão na Seção~\ref{sec:chr}.
A monografia termina com as conclusões na Seção~\ref{sec:con}, seguida de agradecimentos e bibliografia.

\section{Revisão de literatura}
A área das redes complexas é relativamente nova ($\approx 25$ anos)
e a literatura apresenta definições divergentes da área em si.
Uma definição que tem recebido aceitação crescente
é da rede complexa como ``um grafo grande com características topológicas
não triviais''. Esta definição é enganosa ao menos em três pontos.
Primeiro, há redes de interesse com características topológicas triviais, como as redes de Erdös-Rényi e a Geográfica~\cite{newman}, ou as redes simples usadas para exemplos.
Segundo, a definição falha ao não emitir a mensagem fundamental de que uma rede complexa não é somente uma estrutura matemática, um grafo isolado. As redes complexas de interesse são redes reais ou modelos idealizados para as entender.
Além disso, não só grafos grandes são de interesse, mas grafos pequenos são comumente usados como exemplos de propriedades e extensão das estruturas maiores.
Uma definição, ainda longe de perfeita, mas preferida neste trabalho, é considerar a área das redes complexas como interessada em
``redes usualmente grandes, consideradas no, ou para consideração do,
meio em que residem''.
Esta definição resolve ambas as questões.

Os livros em geral apresentam um comum e poderoso repertório para a caracterização de sistemas complexos através de grafos. Talvez as mais notáveis características deste repertório sejam:
\begin{itemize}
	\item O arsenal de medidas: grau, força, betweenness centrality, coeficiente de clusterização, etc.
	\item Os paradigmas básicos de redes: Erdös-Rényi, geográfica, de mundo pequeno e livre de escala.
	\item A abordagem transdisciplinar para considerar o meio no qual a rede está inserida, ou que implica na rede.
\end{itemize}

A literatura sobre análise de redes sociais (ARS, ou \emph{SNA} para \emph{Social Network Analysis}), por exemplo, pode ser frequentemente compreendida como redes complexas em sistemas sociais humanos.

Uma consideração cuidadosa dos livros e artigos lidos para esta pesquisa estão na Seção~\ref{}. As seções a seguir (\ref{sec:jar} e~\ref{sec:misc})
explicitam peculiaridades do jargão da área e considerações sobre as áreas secundárias.

\section{Ambiguidades e sinônimos no jargão}\label{sec:jar}
A área de redes complexas é recente e conflui com diversas 
correntes científicas, como a física, a biologia e a sociologia.
Portanto, possui termos ambiguos e sinònimos.

Exemplos de ambiguidade, sinônimos e delimitações adotadas:
\begin{itemize}
	\item Os vértices mais conectados são, por definição, chamados hubs da rede. O vértice mais conectado é chamado hub da rede. No contexto do algoritmo HITS, o que é bem comum, estes significados mudam: os hubs são os que possuem mais arestas saindo (grau de saída); as autoridades recebem as arestas, ou são referenciados por vários hubs e outras entidades.
	\item Há uma definição de centro e periferia com relação ao raio e diâmetro da rede~\cite{newman,networkX}.
		Por extensão os intermediários podem ser considerados os que não são centro nem periferia.
		Esta setorialização centro, intermediários e periferia gera frações que diferem do previsto pela literatura para as frações de hubs, intermediários e periféricos.
		Um método apropriado para realizar esta setorialização da rede, com resultados estáveis e significativos, consta na Seção~\ref{sec:set}.
	\item \emph{Aresta} e \emph{ligação} são usadas como sinônimos. \emph{Nó} e \emph{vértice} também. É comum o uso de outros termos, em geral coerentes com a aplicação, como \emph{agente}, \emph{ator} ou \emph{participante} para vértices de redes observadas em sistemas humanos.
\end{itemize}

\subsection{Processamento de linguagem natural, dados ligados, participação social}\label{sec:misc}
Diversos títulos foram lidos sobre processamento de linguagem natural, mineração de texto,
visualização de dados e web semântica. Estas áreas tem impacto sobre o que está feito, e sendo feito,
e foram cursadas formalmente uma disciplina sobre cada uma para o doutorado.
Seguem informações pontuais sobre cada área.

Os termos processamento de linguagem natural (PLN) e mineração de
texto (MT) podem em geral serem substituídos um pelo outro.
O termo PLN é preferido nesta pesquisa pois o intuito é mais confluente: 
compreender como a linguagem verbal está
sendo usada para significar.

Os termos web semântica e dados ligados em geral também podem ser 
substituídos um pelo outro. O primeiro salienta
a rede de referenciamento dos dados,
o segundo os dados referenciando-se. Principalmente na esfera
acadêmica, a área é, salvo segunda ordem, sinônimo de dados em 
RDF via XML ou Turtle, ontologias OWL e máquinas de inferència.

A visualização de dados de grafos em evolução temporal é bastante
incipiente. Os poucos casos da literatura foram visitados. As
animações abstratas de redes em evolução, e as ``audiovisualizações''
das redes, que disponibilizamos como parte desta pesquisa,
são potencialmente contribuições na fronteira da visualização.
Video, porém, não é o formato mais apreciado pela literatura
de visualização de dados, que tende a qualificar as figuras
bidimensionais como as mais apropriadas para a pesquisa
científica.

A participação social é a incorporação da própria sociedade nos 
processos de governança da sociedade.
Quase toda a participação social atual é indireta e presencial, 
com a população fornecendo diretrizes, indicadores e acompanhamento
para o setor público.
A participação social tem sido fortalecida no mundo todo,
e conceitos como transparência, participação direta (participação direta 
da sociedade civil na tomada de decisões pelo Estado) e
democracia líquida (atribuição recursiva de competência para tomada
de decisão), se estabelecendo aos poucos como diretrizes para
governos, acadêmicos e sociedade civil.

\chapter{Materiais}\label{sec:mat}
\section{O banco Gmane de dados públicos sobre listas de email (benchmark)}

Mensagens de lista de email foram obtidas do arquivo Gmane~\cite{gmanePack},
que consiste em mais de 20 mil listas de email e mais de 130 milhões de mensagens~\cite{GMANEwikipedia}.
Estas listas cobrem uma variedade de assuntos, em especial
relacionados à tecnologia. O arquivo pode ser descrito como
um corpus com metadados de emails, que incluem hora e lugar
de envio, nome e email do remetente. O uso do GMANE para
pesquisa científica é incidente no estudo de listas isoladas
e de inovações lexicais~\cite{GMANE2,bird}. 

\section{Facebook, Twitter, Participa.br, Cidade Democrática, AA}
Embora as redes de email tenham sido usadas como referência
na observação de propriedades gerais, outras fontes
foram analisadas:

\begin{itemize}
	\item Redes de amizade e interação do Facebook. 8 são usadas como referência em~\cite{timeS}, mas dezenas, talvez algumas centenas, foram observadas nos experimentos da Seção~\ref{sec:exp}.
	\item Milhares de tweets (talvez alguns milhões), geralmente vinculados à alguma \emph{hashtag}. Em especial, a rede de retweets de 22 mil tweets com a hashtag \#arenaNETmundial, foi analizada em~\cite{timeS}.
	\item Mecanismos participativos como o Participa.br, Cidade Democrática e o AA. As redes de amizade e de interação do Participa.br foram analizadas em~\cite{timeS}.
\end{itemize}

\chapter{Métodos}
Para realização desta pesquisa, foram
necessários métodos consagrados, adequações
e variantes.
Esta seção expõe uma seleção destes métodos,
para organizar o conhecimento
e exemplificar esta diversidade:
\begin{itemize}
	\item A Seção~\ref{sec:cir} expõe medidas simples
de estatística circular, ou direcional.
A contribuição neste caso é unicamente nos
padrões encontrados, o método é bastante estabelecido.
	\item A Seção~\ref{sec:intNet} expõe a síntese de redes
		de interação. Talvez haja contribuição na síntese
		do conceito de redes de interação, pois não 
		encontramos (ainda) na literatura tal exposição
		concisa. De qualquer forma, o conceito e o procedimento
		para obtenção das redes a partir de dados é usual,
		a exposição neste texto e no artigo~\cite{timeS} serve
		principalmente ao intuito de formalização do
		processo.
	\item A Seção~\ref{sec:sec} é dedicada ao ``Seccionamento de Erdös'', para obtenção dos três setores básicos da rede, compostos por: hubs, intermediários e periféricos. O método parece não ter sido aplicado antes para este fim, e é resultado imediato da observação das caudas longas de dados reais contrastadas com a rede Erdös-Rényi~\cite{3setores}.
\end{itemize}


\section{Estatística temporal e circular}

Para observação de padrões temporais, foram consideradas escalas diferentes.
Em cada escala, de segundos e meses, foram construídos histogramas de atividade
e feitas algumas medidas de estatística circular.
A fração $\frac{b_h}{b_l}$ entre a maior $b_h $ e a menor $b_l$ incidência nos histogramas
serviram como pista sobre quão uniforme são as distribuições observadas.

Considere cada $ medida $ (dado pontual) como um número complexo com módulo 1, $z=e^{i\theta}=\cos(\theta)+i\sin(\theta)$, onde $\theta= medida \frac{2\pi}{periodo}$. Os momentos $m_n$, tamanhos dos momentos $R_n$, ângulo médio $\theta_\mu$, e o ângulo médio reescalado $\theta_\mu'$ são definidos assim:
\begin{align}\label{eq:cmom}
	m_n&=\frac{1}{N}\sum_{i=1}^N z_i^n \nonumber\\
	R_n&=|m_n|\\
	\theta_\mu&=Arg(m_1) \nonumber \\
	\theta_\mu'&=\frac{period}{2\pi} \theta_\mu \nonumber
\end{align}

$\theta_\mu'$ é usado como medida de localização. A dispersão é medida usando a variância circular $Var(z)$, 
o desvio padrão circular $S(z)$, e a dispersão circular $\delta(z)$:

\begin{align}\label{eq:cmd}
	Var(z)&=1 - R_1 \nonumber\\
	S(z)&= \sqrt{-2\ln(R_1)}\\
	\delta(z)&=\frac{1-R_2}{2 R_1^2} \nonumber
\end{align}

Como esperado, há uma correlação positiva entre $Var(z)$, $S(z)$ e $\delta(z)$,
como pode ser notado nas informações de suporte de~\cite{timeS}.
A medida $\delta(z)$ foi preferida na discussão dos resultados.

\subsection{Formação das redes de interação}\label{sec:intNet}
Redes de interação podem ser modeladas tanto com quanto sem peso, tanto dirigida quando não dirigida~\cite{bird,newmanCommunityDirected,newmanCommunity2013,newmanBook}.
Neste trabalho, quando possível, consideramos redes dirigidas e com peso, a mais informativa das possibilidades.
Nestes casos, desconsideramos as versões dirigidas sem peso, não dirigidas com peso e não dirigidas e sem peso.

Em geral, as redes de interação são obtidas da seguinte forma:
uma reação direta do participante B a uma mensagem do participante A implica em uma aresta de A para B,
representando a informação que foi de A para B.
O raciocínio é: se B reagiu
a uma mensagem de A, ele/ela leu o que A escreveu e formulou uma reação, portanto B
assimilou informação de A, assim $A \rightarrow B$.
A inversão da direção da aresta produz a rede de status: B leu a mensagem e considerou
o que A escreveu digno de resposta, dando status para A, portanto $B \rightarrow A$.
Neste trabalho, as redes de interação são dirigida conforme o fluxo de informação, $A \rightarrow B$.
A Figura~\ref{formationNetwork} expõe esta formação. Maiores detalhes são:arestas em ambas as direções são consideradas distintas;
selfloops são consideradas não informativas (para os interesses atuais) e descartadas;  
a primeira interação $A\rightarrow B$ cria a aresta com peso um; 
a cada nova interação $A\rightarrow B$ um é adicionado ao peso da aresta.
Estas redes de interação humana constam na literatura como portadoras
de propriedades livres de escala (e pequeno mundo), como esperado
para (algumas) redes sociais~\cite{bird,newmanBook}.

\begin{figure}[!h]
	\centering
	\includegraphics[width=0.5\textwidth]{figs/criaRede__}
	\caption{A formação da rede de interação a partir
		de mensagens e respostas.
		Cada vértice representa um participante.
		Uma resposta do participante B a uma
		mensagem do participante A é considerada
		evidência de que B recebeu informação de A,
		representada então por uma aresta dirigida.
		Múltiplas mensagens adicionam ``peso'' à
		aresta dirigida. Maiores detalhes
		estão na Seção~\ref{sec:intNet}}
	\label{formationNetwork}
\end{figure}

\section{Seccionamento de Erdös}\label{sec:sec}
Em uma rede livre de escala, os setores periféricos, intermediários
e de hubs podem ser observados através de uma comparação
com uma rede de Erdös Rényi com o mesmo número de arestas e vértices~\cite{3setores}, como na Figura~\ref{fig:setores}.
Referiremos-nos a este procedimento como \emph{seccionamento de Erdös},
com os setores resultantes chamados \emph{setores de Erdös} (ou \emph{setores primitivos}, \emph{setores básicos} da rede).

A distribuição de grau
$\widetilde{P}(k)$
de uma rede livre de escala ideal
$\mathcal{N}_f$ com $N$ 
vértices e $z$ arestas possui menos
vértices com grau médios do que a distribuição $P(k)$
de uma rede Erdös-Rényi com o mesmo número de vértices e arestas.
De fato, definimos (neste trabalho) o setor intermediário de uma
rede como sendo o conjunto de todos os vértices cujo grau é
menos abundante em uma rede real do que no modelo de Erdös-Rényi:
\begin{equation}\label{criterio}
	\widetilde{P}(k)<P(k) \Rightarrow \text{k é grau intermediário}
\end{equation}

Se $\mathcal{N}_f$ for dirigida e não possuir selfloops, a probabilidade
de existência de uma aresta entre dois vértices arbitrários é
$p_e=\frac{z}{N(N-1)}$.
Um vértice em um dígrafo de Erdös-Rényi com o mesmo número de vértices e aresta,
portanto mesma probabilidade
$p_e$
para existência de aresta, terá grau $k$ com probabilidade:
\begin{equation}
	P(k)=\binom{2(N-1)}{k}p_e^k(1-p_e)^{2(N-1)-k}
\end{equation}

A cauda longa de graus baixos consiste nos vértices de borda,
i.e. o setor periférico ou periferia, onde
$\widetilde{P}(k)>P(k)$ e $k$ é mais baixo
que qualquer valor intermediário de $k$.
A cauda londa de grau alto é o setor dos hubs,
i.e.
$\widetilde{P}(k)>P(k)$ e $k$ é maior que qualquer valor de $k$ do setor
intermediário.
O raciocínio para esta classificação é: os vértices tão conectados
que são virtualmente inexistentes em redes conectadas por puro acaso
(i.e. sem ligação preferencial) são corretamente associadas
aos hubs.
Vértices com pouquíssimas conexões, e muito mais abundantes do que esperado
por puro acaso, são atribuídos à periferia.
Vértices com valores de grau previstos como os mais abundantes caso
as conexões sejam fruto de puro acaso, valores próximos da média,
e menos abundantes em nas redes reais, são classificados como intermediários.


\begin{figure}[!h]
	\centering
	\includegraphics[width=0.5\textwidth]{figs/fser_}
	\caption{As distribuições de grau de modelos ideais de redes livres de escala e Erdös-Rényi
		A segunda possui mais vérices intermediários, enquanto a primeira possui mais vértices
		periféricos e hubs. As bordas dos setores são definidas pelas duas intersecções $k_L$
		e $k_R$ das distribuições de conectividade.
		Os graus característicos estão nos intervalos compactos:
                $[0,k_L]$, $(k_L,k_R]$, $(k_R,k_{max}]$
	para os setores de Erdös (periferia, intermediários e hubs).}	
		\label{fig:setores}
\end{figure}

Para assegurar a validade estatística dos histogramas, os intervalos podem
ser escolhidos de forma que contenham ao menos $\eta$ vértices da rede real.
Assim, cada intervalo, começando no grau $k_i$, estende-se por
$\Delta_i=[k_{i},k_{j}]$,
onde $j$ é o menor inteiro tal que há ao menos $\eta$ vértices com grau maior que ou igual a $k_i$,
e menos que $k_j$. Isso altera a equação~\ref{criterio} para:

\begin{equation}\label{criterio2}
	\sum_{x=k_i}^{k_j} \widetilde{P}(x) < \sum_{x=k_i}^{k_j} P(x) \Rightarrow \text{i é intermediário}
\end{equation}

Se a força $s$ for usada para comparação, $P$ permanece a mesma, mas
$P(\kappa_i)$ com $\kappa_i=\frac{s_i}{\overline{w}}$
deve ser usado na comparação, com
$\overline{w}=2\frac{z}{\sum_is_i}$
o peso médio da aresta e $s_i$ o peso do vértice $i$.
Para graus de entrada e saída
($k^{in}$, $k^{out}$)
a comparação com a rede real deve ser feita com:
\begin{equation}
	\hat{P}(k^{way})=\binom{N-1}{k^{way}}p_e^k(1-p_e)^{N-1-k^{way}}
\end{equation}

\noindent where \emph{way} (sentido) pode ser \emph{in} or \emph{out} (entrada e saída).
Forças de entrada e saída ($s^{in}$, $s^{out}$)
são divididas por
$\overline{w}$
e comparadas também usando $\hat{P}$. Note que $p_e$ permanece a mesma,
pois cada aresta é uma aresta de entrada (ou de saída), e há no máximo $N(N-1)$
arestas entrando (ou saindo), portanto
$p_e=\frac{z}{N(N-1)}$
assim como no caso do grau total

Em outras palavras, seja $\gamma$ e $\phi$ inteiros no intervalos.
$1 \leq \gamma \leq 6$, $1 \leq \phi \leq 3$,
e cada uma das seis possibilidades de seccionamento de Erdös 
$\{E_{\gamma}\}$
possui três setores de Erdös
$E_{\gamma}= \{e_{\gamma, \phi} \}$
definidos como:

\begin{alignat}{3}\label{eq:part}
	e_{\gamma,1}&=\{\;i\;|\;\overline{k}_{\gamma,L}\geq&&\overline{k}_{\gamma,i}\} \nonumber \\
	e_{\gamma,2}&=\{\;i\;|\;\overline{k}_{\gamma,L}<\;&&\overline{k}_{\gamma,i}\leq\overline{k}_{\gamma,R}\} \\ 
	e_{\gamma,3}&=\{\;i\;|\;&&\overline{k}_{\gamma,i}<\overline{k}_{\gamma,R}\} \nonumber
\end{alignat}

\noindent onde $\{\overline{k}_{\gamma,i}\}$ é:

\begin{equation}
	\begin{split}
		\overline{k}_{1,i}&=k_i \\
		\overline{k}_{2,i}&=k_i^{in} \\
		\overline{k}_{3,i}&=k_i^{out} \\
		\overline{k}_{4,i}&=\frac{s_i}{\overline{w}} \\
		\overline{k}_{5,i}&=\frac{s_i^{in}}{\overline{w}} \\
		\overline{k}_{6,i}&=\frac{s_i^{out}}{\overline{w}} \\
	\end{split}
\end{equation}

\noindent e ambos $\overline{k}_{\gamma,L}$ e $\overline{k}_{\gamma,R}$ são encontrados  usando
$P(\overline{k})$ ou $\hat{P}(\overline{k})$
como descrito acima.

Como métricas diferentes podem ser usadas para identificas
os três tipos de vértices, critérios compostos podem ser
definidos. Após uma inspeção cuidadosa das possibilidades,
os critérios compostos foram reduzidos a 6: $\{C_\delta\}_{\delta=1}^{6}$.
Utilizando as Equações~\ref{eq:part}, estes critérios compostos $C_\delta$, com $\delta$ inteiro no intervalo $1\leq\delta<6$ podem ser descritos como:

\begin{equation}
	\begin{split}
		%\begin{multline}
		C_1&=\{c_{1,\phi}=\left\{i\mid i\;\in e_{\gamma,\phi}, \;\forall\; \gamma\}\right\} \\
		C_2&=\{c_{2,\phi}=\left\{i\mid \exists \;\;\gamma: i \in e_{\gamma,\phi}\}\right\} \\
		C_3&=\{c_{3,\phi}=\left\{i\mid i\;\in e_{\gamma,\phi'}, \;\forall\; \gamma,\;\forall\;\phi'\geq \phi\}\right\} \\
		C_4&=\{c_{4,\phi}=\left\{i\mid i\;\in e_{\gamma,\phi'}, \;\forall\; \gamma,\;\forall\;\phi'\leq \phi\}\right\} \\
		C_5&=\{c_{5,\phi}=\left\{i\mid i\;\in e_{\gamma,\phi'}, \;\forall\; \gamma,\right.\\
																																	&\;\;\;\;\;\;\;\;\;\;\;\;\;\;\;\;\;\; \left.\;\forall\;(\phi'+1)\%4\leq (\phi+1)\%4\}\right\} \\
		C_6&=\{c_{6,\phi}=\left\{i\mid i\;\in e_{\gamma,\phi'}, \;\forall\; \gamma,\right.\\
																																	&\;\;\;\;\;\;\;\;\;\;\;\;\;\;\;\;\;\; \left.\;\forall\;(\phi'+1)\%4\geq (\phi+1)\%4\}\right\} \\
		%\end{multline}
	\end{split}
\end{equation}


No artigo~\cite{timeS}, os critérios $C_1$, $C_3$ e $C_5$ foram chamados exclusivistas, os critérios $C_3$ e $C_4$ de cascata e os critérios $C_5$ e $C_6$ de externos. Note que uma cascata exclusivista $C_3$ é a mesma classificação que uma cascata invertida (considera-se dos periféricos aos hubs) e inclusivista. Estes critérios compostos são especialmente úteis para observar estruturas com poucos participantes ou fruto de pouca atividade (veja as figuras do documento de Supporting Information de~\cite{timeS}).

\section{Média e desvio do PCA ao longo do tempo}
A Análise de Componentes Principais (PCA é a sigla consagrada, do inglês Principal Component Analysis) foi usada para observar a estabilidade
na formação das componentes principais.
A PCA é bastante estabelecida e bem documentada
e foi usado para saber: 
1) quais as medidas que contribuiem para cada componente e em que proporção;
2) quanto da dispersão está concentrada em cada componente.

Ou seja, foram analizados os autovetores e autovalores das matrizes
de vértices e suas medidas da seguinte forma: seja $\mathbf{X}=\{X[i,j]\}$
a matriz de todos os vértices $i$ e respectivos valores de cada medida $j$, 
$\mu_X [j]=\frac{\sum_j X[j]}{J}$ a média da métrica $j$, 
$\sigma_X [j]=\sqrt{\frac{(X[j]-\mu_X [j])^2}{J}}$ o desvio padrão da métrica $j$,
e $\mathbf{X'}={\frac{X[i,j]-\mu_X[j]}{\sigma_X[j]}}$ a matriz com \emph{z-score} de cada métrica $j$ de de $\mathbf{X}$ em cada coluna. Seja $\mathbf{V}=\{V[j,k]\}$ a matriz $JxJ$ de autovetores da matriz $\mathbf{C}$ de covariância
de $\mathbf{X'}$, um autovetor por coluna.
Cada autovetor combina as medidas originais em uma componente principal, portanto, basta observar 
$V'[j,k]=100*\frac{|V[j,k]|}{\sum_{j'} |V[j',k]|}$
para saber com que percentagem a medida $j$ contribuiu
para a componente principal $k$.
Com o vetor de $k$ autovalores $D[K]$,
basta observar $D'[k]=100*\frac{D[k]}{\sum_{k'}D[k']}$ para saber
a percentagem da dispersão pela qual a componente principal é responsável.
Com os autovalores k ordenados de forma decrescente, 
em geral basta observar os primeiros três autovalores e respectivos
autovetores em percentagens $\{(V'[j,k],\;D'[k])\}$, pois em geral
já revelam padrões suficientes para uma boa análise e somam entre 60 e 95\% da dispersão de todo o sistema.
Em~\cite{timeS}, em especial, foram feitas médias e desvios das constribuições de cada componente para a dispersão e das medidas em cada componente. Ou seja, dadas $L$ observações $l$, cada uma com $k$ pares de autovalores e autovetores, são observadas, para cada medida,
a média $\mu_{V'}[j,k]$ e desvio $\sigma_{V'}[j,k]$ 
da medida j na componente principal k,
e a média $\mu_{D'}[k]$ e desvio $\sigma_{D'}[k]$ da 
contribuição da componente $k$ na dispersão do sistema:

\begin{align}\label{eq:pca}
\mu_{V'}[j,k]   &=\frac{\sum_l^L V'[j,k,l]}{L}\nonumber\\
\sigma_{V'}[j,k]&=\sqrt{\frac{(\mu_{V'}-V'[j,k,l])^2}{L}}\\\nonumber
\mu_{D'}[k]&=\frac{\sum_l^L D'[k,l]}{L}\\\nonumber
\sigma_{D'}[k]&=\sqrt{\frac{(\mu_{D'}-D'[k,l])^2}{L}}
\end{align}

A matriz de covariância $\mathbf{C}$ também é observada diretamente para
uma primeira pista sobre os padrões. Isso é feito com associações simples: valores absolutos pequenos indicam baixa correlação (a princípio independência); valores altos indicam correlação positiva (diretamente proporcional);
valores negativos com módulo grande indicam correlação negativa (inversamente proporcional).

\subsection{Medidas consideradas e acrescentadas}\label{sec:med}
A topologia das rede sestudas foram estudadas utilizando PCA~\cite{pca}
com uma pequena seleção das medidas mais básicas e fundamentais de cada vértice. Formalmente, sejam $i$, $j$ vértices e $e_{ij}$ uma aresta de $j$ para $i$ (ou $j\rightarrow i$) e $w_{ij}$ seu peso. Então:

\begin{itemize}
	\item Grau $k_i=\sum_j (e_{i,j}+e_{j,i})$: número de arestas conectadas a $i$.
	\item Grau de entrada $k_i^{in}=\sum_j e_{i,j}$: número de arestas que terminam no vértice $i$.
	\item Grau de saída $k_i^{out}=\sum_j e_{j,i}$: número de arestas que partem do vértice $i$.
	\item Força $s=\sum_j (w_{i,j}+w_{j,i})$: soma dos pesos de todas as arestas conectadas ao vértice $i$.
	\item Força de entrada $s_i^{in}=\sum_j w_{i,j}$: soma dos pesos de todas as arestas que terminam no vértice $i$.
	\item Força de saída $s_i^{out}$: sima dos pesos de todas as arestas que partem do vértice $i$.
	\item Coeficiente de clusterização $cc_i=\frac{\sum e_{j_1 j_2}}{\binom{k_i}{2}}$: fração de pares de vizinhos $j_1$, $j_2$ de $i$ que são conectados. A medida usual para grafos não direcionados foi usada.
\item Intermediação (betweenness centrality) $bt_i=\frac{\Delta_i}{\Delta}$: fração entre o número $\Delta_i$ de geodésicas entre cada par de vértices da rede que contém o vértice $i$ e $\Delta$, o número total de geodésicas entre cada par de vértices da rede.
	A intermediação foi calculada considerando direções e peso, como especificado em~\cite{faster}.
\end{itemize}

Para apreender as simetrias das atividades dos participantes, as
seguintes métricas foram introduzidas para o vértice $i$:

\begin{itemize}
	\item Assimetria: $asy_i=\frac{k_i^{in}-k_i^{out}}{k_i}$.
	\item Média da assimetria das arestas: $\mu_i^{asy}=\frac{\sum_{j\in J_i} e_{ji}-e_{ij}}{|J_i|=k_i}$, onde $e_{xy}$ é 1 se houver aresta de $x$ para $y$, e $0$ caso contrário. $J_i$ é o conjunto de vizinhos do vértice $i$, e $|J_i|=k_i$ é o número de vizinhos do vértice $i$.
	\item Desvio padrão da assimetria das arestas: $\sigma_i^{asy}=\sqrt{\frac{\sum_{j\in J_i}[\mu_{asy} -(e_{ji}-e_{ij}) ]^2  }{k_i}  }$.
	\item Desequilíbrio: $dis_i=\frac{s_i^{in}-s_i^{out}}{s_i}$.
	\item Média do desequilíbrio das arestas: $\mu_i^{dis}=\frac{\sum_{j \in J_i}\frac{w_{ji}-w_{ij}}{s_i}}{k_i}$, onde $w_{xy}$ é o peso da aresta $x\rightarrow y$ e zero se não houver tal aresta.
	\item Desvio padrão do desequilíbrio das arestas: $\sigma_i^{dis}=\sqrt{\frac{\sum_{j\in J_i}[\mu_{dis}-\frac{(w_{ji}-w_{ij})}{s_i}]^2}{k_i}}$.
\end{itemize}

\section{Teste de Kolmogorov-Smirnoff para os textos produzidos por cada setor}

Sejam $F_{1,n}$ e $F_{2,n'}$ duas distribuições cumulativas empíricas onde $n$ e $n'$ são o número de observações em cada amostragem.
O teste de Kolmogorov-Smirnov de amostragem dupla 
rejeita a hipótese nula se:
\begin{equation}\label{eq:ks}
D_{n,n'} > c(\alpha)\sqrt{\frac{n+n'}{nn'}}
\end{equation}

onde $D_{n,n'}=sup_x[F_{1,n}-F_{2,n'}]$ e $c(\alpha)$ é dado para cada $\alpha$ segundo a Tabela~\ref{tab:kol}:

\begin{table}[!h]
\centering
\caption{Relação entre $\alpha$ e $c(\alpha)$ para o teste de Kolmogorov-Smirnov}\label{tab:kol}
\begin{tabular}{|l||c|c|c|c|c|c|}\hline
$\alpha$    & 0.1  & 0.05 & 0.025 & 0.01 & 0.005 & 0.001 \\\hline
$c(\alpha)$ & 1.22 & 1.36 & 1.48  & 1.63 & 1.73  & 1.95  \\\hline
\end{tabular}
\end{table}



\subsection{Adaptação}

Para alguns dos resultados, 
utilizamos $c(\alpha)$ como pista sobre
o quão diferentes são pares de distribuições empíricas.
São calculados $D_{n,n'}$, enquanto $n$ e $n'$ são dados.
Assim, todos os termos da Equação~\ref{eq:ks} são positivos
e $c(\alpha)$ pode ser isolado:

\begin{equation}\label{eq:ks2}
c(\alpha) < \frac{D_{n,n'}}{\sqrt{\frac{n+n'}{nn'}}} = c'(\alpha)
\end{equation}

\section{Audiovisualização de dados}
Redes foram visualizadas com imagens, videos e engenhocas online para esta pesquisa~\cite{animacoes,galGMANE,appGMANE}. Também foram sonificadas, em especial como faceta auditiva de animações abstratas~\cite{preludio,4hubs,gmane,social}.
Tais ``audiovisualizações'' foram cruciais para guiar a pesquisa para
as características mais importantes das redes de evolução.
Além disso, os tamanhos relativos dos três setores de Erdös foram visualizados como linhas temporais.
A visualização da estrutura em rede foi especialmente útil na inspeção
dos dados e estruturas das redes de email.

\section{Considerações tipológicas e humanísticas}
As redes estudadas são constituídos de seres humanos.
Quando há classificação envolvida, sejam dos agentes
ou dos sistemas em si, reflexões humanísticas são pertinentes, como as disparadas pelas perguntas:

\begin{itemize}
	\item Qual o potencial estigmatizante da classificação?
	\item O que mais sabemos sobre o indivíduo ou a rede que é classificada, ou seja, é considerada(o) de um tipo?
	\item Quais dados posso usar sem desviar a atenção para leis
		e processos de comitês de ética?
	\item Qual a melhor forma de proceder com os dados e conhecimentos relacionados?
\end{itemize}

Todas estas questões, e muitas outras, estão em constante amadurecimento com grupos de pesquisa~\cite{Nexus}, leituras, e contatos individuais com outros pesquisadores~\cite{massimoTexto,massimoEncontro}.

\section{Web semântica}

Para a formalização de conceitualizações, e
para formatos de dados apropriados para armazenamento,
compartilhamento e referência,
foram adotadas as recomendações de dados ligados / web semântica
da W3C~\cite{rdf,owl,rdfs}.
De forma bastante resumida, o arcabouço utilizado pode
ser viso como uma forma de formalizar conceitos (classes), relações
entre conceitos (propriedades) e instâncias dos conceitos (indivíduos).
As informações são expressas de forma semi-estruturada em RDF:
triplas ``sujeito predicado objeto'',
com o sujeito sempre uma classe, o predicado sempre uma propriedade,
e o objeto sempre uma classe ou dado. As propriedades podem ter aspectos
específicos, chamados ``axiomas de propriedade''. As classes podem ser restritas a possuírem relações arbitrárias, chamadas ``restrições de classe''.
É uma recomendação da W3C e o padrão acadêmico para dados ligados, i.e. para representação na web semântica.

Utilidades da tecnologia incluem:
\begin{itemize}
	\item inferência por máquina através de especificações ontológicas. 
	\item Interconexão de dados de fontes (bases de dados) diferentes.
	\item Organização de conhecimento específico para consideração cuidadosa, seja individual ou em grupo.
\end{itemize}

As ontologias são chave dentre as tecnologias de web semântica.
Uma ontologia é geralmente definida como uma ``especificação de
uma conceitualização'', e a recomendação é o uso do padrão OWL.
Os vocabulários são coleções de termos e metadados, como definição,
e a recomendação é o uso do padrão SKOS.
O estado da arte de web semântica tem apresentado avanços,
por exemplo, é capaz de realizar inferências
úteis, especialmente para buscas. Por outro lado, é uma tecnologia
complicada e com algumas dificuldades de implementação.
Por exemplo, um conceito SKOS é um indivíduo, e uma classe OWL,
se identificado com um conceito SKOS é, por consequência um
indivíduo. Neste caso, os recursos de inferência por máquina 
ficam limitados
dada a complexidade da especificação.
Dito de outra forma, caso na especificação conste uma classe

\subsection{A construção de ontologias OWL e vocabulários SKOS}

Para formalizar conceitualizações referentes às estruturas sociais,
mais especificamente relacionadas à participação social,
foram construídas ontologias OWL e vocabulários SKOS a partir de entrevistas
com especialistas: acadêmicos e gestores públicos.
Também foram feitas ontologias e vocabulários a partir de bancos de dados,
decretos presidenciais e outras documentações.
O processo  consistiu sempre que possível na coleta de informações,
formalização dos conceitos e devolutiva aos entrevistados,
com figuras e outras documentações, até que não tivessem
mais contribuições.

\subsection{A triplificação de dados relacionais}
Para disponibilização e uso de dados de diferentes fontes, foram
feitos pequenos programas de computador (scripts) para
acessar dados relacionais e escrever triplas RDF. Estes scripts
criam conceitos novos e os vincula às instâncias dos dados.
Na sequência, acessa as ontologias pertinentes, salva
uma versão com os dados e ontologias, e uma versão
com os dados, as ontologias e as triplas resultantes da inferência.


\chapter{Resultados}
%\section{Time stability in human interaction networks}
\section{Estabilidade temporal e diferenciação textual em redes de interação humana}\label{sec:timeS}
Explicitados cuidadosamente em~\cite{timeS}, os principais resultados são:
\begin{itemize}
	\item A atividade ao longo do tempo é praticamente a mesma para todas as listas de email analisadas, e em todas as escalas. A maior dispersão foi encontrada nos segundos e minutos, seguida pelos dias do mês, meses, dias da semana e horas do dia. Padrões estáveis foram apreciados em todas estas escalas: segundos, minutos e dias do mês apresentaram uniformidade; meses parecem seguir calendários acadêmicos e escolares; dias da semana apresentam redução para dois ou um terço das atividades nos finais
		de semana;
		nas horas do dia, há concentração de atividades das 12-18h, o pico, porém, ocorre pouco antes das 12h.
	\item A fração de participantes em cada setor de Erdös é estável ao longo do tempo e pode ser determinada mesmo com poucas mensagens.
	\item As métricas topológicas se combinam nas componentes principais do PCA da mesma forma para todas as listas e todos os snapshots.
	\item As medidas de simetria da topologia, como definidas na Seção~\ref{sec:med}, apresentam mais dispersão do que o usual coeficiente de clusterização. O coeficiente de clusterização se combina com os desvios padrões de assimetria e desequilíbrio para a formação da terceira componente.
	\item Estes comportamentos são muito estáveis para em redes de interação de email. Nas outras redes analisadas, Twitter e Participa.br
		apresentaram redes bastante similares às de email. Nas
	redes do Facebook foram encontradas algumas redes que diferiam
	do modelo apresentado pelas redes de listas públicas de email.
\item Para um mesmo número de mensagens (sejam 20 mil)  e diferentes listas, há uma
	correlação negativa entre número de participantes e número de threads quando os participantes são poucos (até 2 mil participantes quando são 20 mil mensagens). Para uma quantidade maior de participantes, há uma correlação positiva entre o número de participantes e o número de threads. Este fato deve estar relacionada a outras características topológicas e textuais da rede e pode servir para uma tipologia das próprias redes.
	\item Especulações humanísticas, especialmente sobre questões
		tipológicas e antropológicas, são seguem imediatamente
		os resultados quantitativos. Em especial, a setorialização
		de Erdös implica em uma tipologia de agentes em redes humandas de interação.
		Esta tipologia é, a princípio, não estigmatizante pois os agentes mudam de setor o tempo todo. Além disso, um mesmo agente pertence
		a todos os setores ao mesmo tempo, mas em redes diferentes. Maiores qualificações desta tipologia, decorrente do pertencimento
		a um setor de Erdös, estão no proprio artigo.
\end{itemize}

Com base nestes resultados, foi investigada a produção de texto na rede,
com foco na potencial relação entre topologia, setor de Erdös e texto produzido~\cite{rcText}. As principais conclusões são:
\begin{itemize}
	\item Os textos produzidos por cada setor de Erdös diferem bastante: os $c(\alpha)$ fruto das comparações são tão grandes que as tabelas não registram os valores. Além disso, as diferenças entre setores iguais de redes diferentes são, via de regra, maiores que as encontradas entre setores diferentes da mesma rede.
	\item As características topológicas e textuais de cada agente apresentam correlações não triviais (como entre intermediação e uso de advérbios) e triviais (como entre grau e número de caractéres escritos).
		Mesmo assim, são muito menos correlacionadas entre si do que separadamente. Ou seja, as componentes principais possuem tendência
		a prevalência de medidas topológicas {\bf ou} textuais,
		mas combinam-se medidas de ambos os tipos.
	\item Dada uma quantidade grande de texto, algumas medidas estatísticas 
		relacionadas ao tamanho das palavras parecem se conservar. Não foi dada uma explicação definitiva para este fenômeno, embora haja esperança de que consigamo explicações simples (o problema não parece difícil). Algumas destas medidas estão ao final de~\cite{rcText}, resultantes de somas cumulativas de diferencas de histogramas.
\end{itemize}

Estes resultados permearam várias outras frentes de
de pesquisa e desenvolvimento tecnológico~\cite{prod1,prod2,prod3,prod4,prod5,ops,versinus}.

\section{Criação da nuvem brasileira de dados participativos ligados}
Iniciada para formalizar os tipos de rede e participantes, e estruturas
que pudessem formar, esta frente rapidamente se voltou para
formalizações de conceitualizações referentes às estruturas
e sistemáticas já em prática e previstos em lei.
Dados também foram publicados, fazendo uso das ontologias
feitas. Estes dados e ontologias foram em grande parte
já publicados e estão em uso, mas a grande maioria não
recebeu artigo científico.

%\subsection{Published linked data and OWL ontologies}
\subsection{Síntese de ontologias OWL}
Ontologias OWL feitas neste trabalho:
\begin{itemize}
		\item OPS (Ontologia de Participação Social): é uma ontologia fruto de diversos esforços da américa latina, principalmente do Brasil. Nesta pesquisa, revisamos a ontologia e disponibilizamos um código OWL em uso por instâncias diferentes da academia, Estado e sociedade civil.
		\item OPa (Ontologia do Participa.br): esta é uma ontologia feita para e partir dos dados do Participa.br (Portal Federal de Participação Social, SG-PR).
		\item OPP (Ontologia de Portais Participativos): esta ontologia foi pensada com a equipe do Participa.br e outros especialistas como esquema geral de portais participativos. Ontologia relativamente complexa, centrada em 3 classes: Participante, Comunidade, Mecanismo Participativo.
		\item Ontologiaa (Ontologia do AA): é uma pequena ontologia para o minimalista AA (Autorregulação Algorítmica), um software para registrar e compartilhar processos intelectuais como para pesquisa e arte~\cite{paaper, ensaaio}.
		\item OCD (Ontologia do Cidade Democrática): é uma ontologia extensa para o portal participativo Cidade Democrática, da sociedade civil. Dado o tamanho da ontologia, o processo de sua construção deu origem ao método de construção de ontologia OWL a partir dos dados, descrito na Seção~\ref{} e utilizado também para a construção da OPa (acima).
		\item OBS (Ontologia da Biblioteca Social): Esta ontologia consiste na verdade em uma coleção de ontologias, uma para cada conceito que precisasse de uma abordagem dedicada, e uma para cada mecanismo ou instância de participação social prevista no Decreto Presidencial nº 8.243 de 23 de maio de 2014, conhecido como decreto da PNPS ou da Política Nacional de Participação Social. Esta ontologia também contou com entrevistas feitas diretamente para este trabalho, além da contribuição de uma oficina na na Secretaria-Geral da
Presidência da República, para explicitar a utilidade
destas formalizações semânticas e coletar informações sobre
diversos mecanismos e instâncias de participação social previstos
em lei e praticados.
		\item VBS (Vocabulário da Biblioteca Social): este
			vocabulário é uma adaptação (com complementos)
			da OBS no formato
			de vocabulário, principalmente para permitir
			uso junto ao DSPACE.
\end{itemize}

As ontologias são todas construídas através de scripts, com exceção da
OPP, feita no Protegé.

\subsection{Formatação de dados ligados a partir de dados relacionais}
Três roteiros de conversão de dados foram feitos para conversão de dados relacionais em dados RDF enriquecidos semanticamente:
\begin{itemize}
	\item Triplificação do Participa.br: 
		foi elaborado e disponibilizado um roteiro para formatação em RDF de dados relacionais do Participa.br (originalmente em PostgreSQL)~\cite{triplificaParticipa}. Estes dados foram depois usados, através de buscas SparQL, para auxiliar na construção da OPa~\cite{opaScript}.
	\item Triplificação dos dados do Cidade Democrática: 
		foi elaborado um roteiro para formatação em RDF de dados 
		do Cidade Democrática (originalmente em MySQL). Estes
		dados foram depois utilizados para auxiliar na construção
		da OCD~\cite{ocdScript}.
	\item Triplificação dos dados do AA:
		foi elaborado um roteiro para formatação em RDF de
		dados do AA encontrados em bancos de dado MySQL
		e MongoDB, e em logs de IRC.
		Esta triplificação é especialmente útil pela
		simplicidade do AA, o que permite experimentar
		abordagens diferentes e escolher a melhor, antes
		de desenvolver ontologias e triplificações mais complicadas.
		Esta foi a única triplificação feita depois da ontologia e não aproveitada para a construção da ontologia.
\end{itemize}

\subsection{Escritos, ontologias e dados publicados}
Artigo sobre a OPS~\cite{ops}. Produtos PNUD/ONU~\cite{pnud3,pnud4,pnud5}.
Dados em RDF do Participa.br, Cidade
Democrática e AA no Datahub.io~\cite{datahub}.
Ontologias~\cite{opa,opp,ops,obs}. Vocabulários~\cite{vbs}.
Scripts para síntese das ontologias e triplificação
de dados~\cite{opa,ops,obs,partT,cdT,aaT}.

\subsection{Método de construção de ontologias orientado aos dados}
Um método de levantamento de ontologia orientado aos dados surgiu, potencialmente útil a todos os portais e software em necessidade de ontologias, e foi responsável por 2 ontologias (OPA dados e OCD).
Resumidaemente, o método consiste em: representar os dados de interesse como RDF; realizar buscas SparQL para construir ontologia trivial com as classes e propriedades encontradas; realizar buscas SparQL para inferir restrições de classe e axiomas de propriedade.

\section{Aparato em software}
Scripts para verificar as estabilidades topológicas e diferenciações textuais em redes humanas estão reunidos em um pacote oficial da linguagem Python~\cite{gmanePack}.
Estão sendo feitos pacotes para organizar com os scripts todos
de triplificação de dados, construção de ontologias e mineração das
estruturas~\cite{participation, percolation}. 
Os dados, classes e propriedades das
ontologias e triplificações estão também disponíveis (em parte)
através das próprias URIs, redirecionadas via purl.org
para um servidor de pesquisa.
Os dados estão em um endpoint SparQL, e scripts para a mineração
destes dados estão disponíveis em interfaces web via um
IPython Notebook. As ontologias estão também disponíveis
na instalação do Webprotegé da Stanford.

\section{Aproveitamento}
A proposta do trabalho é comprovar e desenvolver algumas formas para o participante destas redes se beneficiar delas.
Os resultados das seções anteriores servem a este fim, sendo, porém, mais desenvolvimentos científicos e tecnológicos do que formas imediatas
de beneficio para o participante.
Esta seção é uma investida mais reta nesta direção, tendo como base os métodos e resultados alcançados.

\subsection{Procedimentos de percolação social}\label{sec:exp}
Foram realizados procedimentos cíclicos e procedimentos efêmeros
no tecido social para observar as reações
e testar hipóteses de modificação das estruturas sociais.
Experimentos paradigmáticos:
\begin{itemize}
	\item Em dezembro de 2012 foram iniciados ciclos de
		coleta e difusão de informação sobre
		as redes sociais e o potencial benéfico para
		o indivíduo civil. Duraram meses e redes diferentes
		foram usadas. Foram confirmadas 
		as hipóteses de modificação das
		estruturas sociais para comportar a pesquisa,
		com suporte humano, financeiro e institucional.
		Foram confirmadas hipóteses de modificação
		do tratamento da sociedade sobre o tema.
		Foram confirmadas hipóteses de que seriam
		verificáveis estes resultados tanto em minhas
		interações cotidianas, via praticamente todos os meios.
		Em especial, a minha rede de amizades do Facebook
		foi utilizada (cada vértice é um amigo meu,
		cada aresta indica uma amizade entre eles), e
		amigo por amigo foi acionado,
		dos menos conectados aos mais conectados.
	\item Escrita para os de maior betweenness e de maior closeness, mostrar que não bate para as listas.
\end{itemize}

O segundo experimento foi feito por vários parceiros de pesquisa, por onde pode ser verificado.
O segundo experimento ainda não foi replicado.
É comum após alguma apresentação alguém se prontificar a fazê-lo,
mas isso nunca aconteceu. 
Eu mesmo já me comprometi comigo em replicar o experimento,
mas não aconteceu. Uma tese usual é que haja bloqueios
mentais que impedem a intervenção tão direta na nossa existência
social, ou nosso eu-rede~\cite{latour,ciberiun}.

\subsection{Física antropológica}
Estes experimentos, e outras anotações de dados, são, no escopo deste
trabalho, consideradas questionáveis, potencialmente erradas mesmo,
caso não sejam observadas algumas diretrizes:
\begin{itemize}
	\item estudo das próprias redes como estudo de si.
	\item Uso de anotações de si com a devida atenção para não expor as pessoas desnecessáriamente.
	\item Abertura constante dos procedimentos, dados e códigos.
\end{itemize}
Estas diretrizes são considerações da tradição antropológica,
e, portanto, configuram uma pesquisa de ``física antropológica''.
O termo começou a ser usado no Brasil principalmente por
acadêmicos (físicos, cientistas da computação, filósofos, antropólogos).

\subsection{Sistemas de recomendação para o enriquecimento da navegação semântica de recursos}
O relacionamento semântico de dados e conceitualizações via tecnologias de web semântica são feitos para que os recursos sejam navegáveis à semelhança das usuais páginas HTML da WWW (por isso chama-se {\bf web} semântica).
Mas, ao invés de páginas HTML, recursos em RDF e links via critérios semânticos.
No decorrer desta pesquisa, surgiram possibilidades de enriquecimento
da navegação semântica. Esta recebeu até prova de conceito~\cite{pnud4}:
\begin{itemize}
	\item São geradas estruturas auxiliares: rede de amizades, rede de interação, histograma de radicais (do texto), seleção dos 400 radicais
		mais incidentes para caracterizar o domínio,
		histograma de radicais de cada recurso (postagem, comentário, participante, etc).
	\item O solicitante pode requerer recomendação de recursos
		de qualquer tipo a partir de um recurso de qualquer
		tipo. Pode optar por métodos de recomendação topológico, textual ou híbrido. Pode optar por polaridade de similareidade, dissimilaridade ou mista.
	\item Os métodos são todos explicitados em texto e código
		para o participante. Cada método conta também com um
		registro de potenciais utilidades para o participante.
\end{itemize}

\subsection{Compreensão sobre as estruturas sociais}\label{sec:com}
Praticamente todos os resultados citados acima
são úteis para compreender as estruturas sociais e são
fruto desta tentativa.
Há a intenção de disponibilizar uma introdução às redes complexas
através da instrumentalização do leitor com estes conhecimentos e tecnologias
para exploração de si próprio. Um esboço
consta em~\cite{gradus}.

Um exemplo especial de fundamentação que parece não 
constar na literatura é a constatação de que
a propriedade livre de escala é consequência da distribuição
equânime de recursos pelos setores da rede.
Para apreender este fato, considere uma quantidade fixa $R$ de recursos
que será utilizada para a realização da rede em cada setor.
Considere que, para cada quantidade de recursos $T$, são contadas $f=\frac{R}{T}$ partes de tamanho T, como na Figura~\ref{fig:1T}.

\begin{figure}[!h]
	\centering
	\includegraphics[width=0.5\textwidth]{figs/rt}
	\caption{A curva resultante da divisão de uma mesma quantidade
	R de recursos em $\frac{R}{T}$ partes de tamanho T.
        Utilizada para expor uma potencial causa da ubiquidade da
        propriedade livre de escala.}
	\label{fig:1T}
\end{figure}

Segue que $log(f)=-1 log(T) + C$, $C=log(R)$ uma constante
arbitrária. Uma reta descreve a relação entre $log(f)$ e 
$log(T)$, como na Figura~\ref{fig:1t}.
Os recursos são alocados pelo sistema de forma uniforme,
pois $T\frac{R}{T}=R=constante$.

Considere que $T=T_1 T_2$ 
(e.g. recursos da rede=agentes x tempo de cada agente). 
Neste caso,
$f=\frac{R}{T_1 T_2}$ e
segue que $log(f)=-log(T_1 T_2) +C$. Se $T_1=T_2$, 
$log(f)=-2 log(T_1) + C$, e
$\gamma=2$ como previsto pela literatura.
No exemplo, o tempo alocado
é o tempo dos próprios agentes,
portanto é razoável considerar $T_1=T_2$. 
Possíveis causas para a distorção do valor exato $\gamma=2$
são: propriedades fractais, recursos em número diferente, associações entre os recursos. 

%\chapter{Finished and planed tasks, chronogram}
\chapter{Afazeres e cronograma}
% Colocar tabela com os comparativos
% Referenciar para complementos no texto.

\section{Comparativo de afazeres}\label{sec:afa}
\begin{table}[!h]
\centering
\footnotesize
\caption{Relação de tarefas feitas e por fazer. Há literatura pronta e vários documentos em mãos para serem aprofundados.}\label{tab:afa}
\begin{tabular}{p{1.3cm}||p{7.6cm}|p{7.1cm}}\hline
	& {\bf feito} & {\bf por fazer} \\\hline
	{\bf escrita}      & artigo de estabilidade em redes de interação humana~\cite{timeS};
	artigo sobre a Ontologia de Participação Social~\cite{ops};
	ensaio descrevendo simbióse com PNUD/ONU e SG-PR~\cite{ensaioAA};
	artigo com descrição psicofísica da música no áudio digial~\cite{massa};
	produtos PNUD 3, 4 e 5, descrevendo sistemas de classificação, recomendação, ontologias e triplificações~\cite{pnud3,pnud4,pnud5};
	artigo sobre AA~\cite{paaper}; 
	versões iniciais e rascunhos dos artigos sobre física antropológica~\cite{pa},
	sobre votação contínua por aprovação e participação~\cite{vote},
	sobre diferenças da produção textual nos setores de Erdös~\cite{rcText},
	sobre visualização de redes de interação em evolução temporal~\cite{versinus},
	sobre audiovisualização de redes de interação em evolução temporal~\cite{versinus},
	sobre performance audiovisual via controle coletivo de código e projeção ao vivo~\cite{vivace},
	& publicar artigos no arXiv; repassar produtos PNUD um e dois;
	``Complex Networks Gradus ad Parnassum'', um compêndio de redes complexas que utiliza a existência
	em rede do leitor para instrumentalizá-lo;
	artigo sobre tipologia de agentes humanos em redes de interação;
	versão desenvolvida do escrito sobre física antropológica;
	documentação do pacote Python oficial ``percolation''~\cite{percolation};
	artigo com o método de levantamento de ontologias orientado aos dados;
	artigo sobre os dados participativos ligados brasileiros;
	artigo com os experimentos de coleta e difusão de informação;
	versão final do ensaio do AA~\cite{ensaaio}\\\hline
	{\bf leitura }     & documentação de redes complexas;
			documentação de web semântica;
			amadurecimentos coletivos frutos das difusões de informação;
	               numerosos artigos da wikipédia, protocolos e manuais de software;
		       cursos do coursera, alguns completos;
		       literatura de PLN;
		       literatura de visualização de dados e mineração de dados;
		       artigos, exemplos especiais são~\cite{} & 
		                       estatística e física estatística, talvez manuais de R também;
				       terminar livros referência de redes complexas;
				       absorver uma literatura mínima sobre antropologia;
				       visita à topologia tradicional e teoria de grafos na computação \\\hline\hline
				       {\bf experi-mentos} & experimentos contínuos e efêmeros  & confirmar experimentos contínuos \\\hline
				       {\bf comuni-dade  }& repassados resultados para comunidades estudadas;
				       confirmada permissão dos desenvolvedores
				      do GMANE para utilizar os dados das listas para pesquisa;
				        & repassar às comunidades estudadas um resumo dos resultados, em linguagem mais acessível que os artigos \\\hline
				      {\bf disciplinas} & cursadas disciplinas Introdução ao Processamento de Linguagem Natural,
				       Mineração de dados;
				       Visualização de dados;
				       Introdução à web semântica & -//- \\\hline
				       {\bf considerar banca} & -//-  & preparar apresentação; apresentar e anotar contribuição da banca; conduzir com orientador \\\hline
\end{tabular}
\end{table}

\begin{table}[!h]
\centering
\footnotesize
\caption{Relação de tarefas feitas e por fazer. Há literatura pronta e vários documentos em mãos para serem aprofundados.}\label{tab:afa}
\begin{tabular}{p{1.3cm}||p{7.6cm}|p{7.1cm}}\hline
	& {\bf feito} & {\bf por fazer} \\\hline
				       {\bf software}     & telões de streaming de estruturas sociais;
				       instância de Monitoramento Massivo e Interativo da Sociedade pela Sociedade para Aproveitamento (MMISSA);
				       engenhoras no AARS (A Análise de Redes Sociais) e MyNSA (Monitoring yields Natural Streaming and Analysis);
				       rotinas de triplificação de dados; rotinas de construção de ontologias;
				       rotinas para, dada a rede social, sintetizar música e animação visual sincronizados;
				       rotinas com fundamentos e provas de conceitos para genérica classificação e recomendação de recursos;
				       & finalizar pacotes oficiais da linguagem Python;
				       estação de monitoramento massivo;
				       sistema de navegação semântica enriquecido com recomendação de recursos;
				       \\\hline
				       {\bf dados}        & dados triplificados do Participa.br, do Cidade Democrática, do AA  & revisar dados triplificados;
				       triplificar dados do Facebook, Twitter e listas de email. \\\hline
				       {\bf ontologias e vocabulários}  & OPS, OPa, OPP, OCD, Ontologiaa, OBS e VBS iniciais.  & ontologias e vocabulários revisados; \\\hline
				       {\bf audiovisu-alização}         & prelúdio social; four hubs dance;   & músicas focando em algum dos participantes da rede; mais músicas sobre as redes do Facebook; mais músicas sobre as redes de email; rotinas para fazer animação abstrata sobre rede de interação e mixar com clipe do youtube;
				       sonificação de dados semânticos e renderização de imagens sincronizadas \\\hline
\end{tabular}
\end{table}








\section{Cronograma}\label{sec:cron}


\begin{table}[h]
\begin{center}
  \caption{Cronograma de atividades ao longo dos semestres, descritas na Seção~\ref{sec:cron}.
	  A marcação $\bullet$ indica previsão feita no início do doutorado.
  A marcação [ ] se refere ao relato e previsão, agora no final do $1^{\circ}$ semestre de 2015.
  As principais diferenças do previsto foram: as disciplinas foram terminadas no primeiro ano; a revisão da literatura, os acréscimos aos modelos atuais com o foco no participante da rede, e a implementação computacional, estas três atividades estão sendo realizadas constantemente e devem durara até pouco antes da entrega e defesa da tese. }
\label{tab:cron}
  \begin{tabular}{ | c ||   c | c |     c | c |   c | c |}
    \hline
      & \multicolumn{2}{|c|}{2013} & \multicolumn{2}{|c|}{2014} & \multicolumn{2}{|c|}{2015} \\
    \hline
    Atividade & 1$^{\circ}$ & 2$^{\circ}$ & 1$^{\circ}$ & 2$^{\circ}$ & 1$^{\circ}$ & 2$^{\circ}$ \\
    \hline \hline

    1 & [$\bullet$] & [$\bullet$] & $\bullet$ & $\bullet$ & & \\
    \hline
    2 & [$\bullet$] & [$\bullet$] & [$\bullet$] & [ ] & [ ] & [ ] \\
    \hline
    3 & [ ] & [$\bullet$] & [$\bullet$] & [$\bullet$] & [$\bullet$] & [ ]  \\
    \hline
    4 & [ ] & [$\bullet$] & [$\bullet$] & [$\bullet$] & [$\bullet$] & [$\bullet$]  \\
    \hline
    5 & & & & & [$\bullet$] & [$\bullet$]  \\
    \hline
    6 & [$\bullet$] & [$\bullet$] & [$\bullet$] & [$\bullet$] & [$\bullet$] & [$\bullet$]  \\
    \hline
    7 & [$\bullet$] & [$\bullet$] & [$\bullet$] & [$\bullet$] & [$\bullet$] & [$\bullet$]  \\
    \hline
  \end{tabular}
\end{center}
\end{table}


Este projeto foi inicialmente dividido segundo as etapas a seguir 
e usadas como referência na Tabela~\ref{tab:cron}:

\begin{enumerate}
\item Créditos Obrigatórios: cumprimento dos créditos obrigatórios em disciplinas, exigidas pelo programa de Doutorado do IFSC/USP.

\item Revisão da literatura.

\item Acréscimos aos modelos atuais com o foco no participante da rede.

\item Implementação computacional.

\item Escrita da tese.

\item Escrita e publicação dos resultados em artigos.

\item Trocas com pessoas externas, estabelecimento de colaborações.
\end{enumerate}

Considerações sobre estes itens:
\begin{enumerate}
\item Foram cursadas as disciplinas de Processamento de linguagem natural, Mineração de dados, Visualização de dados e Web semântica. Dediquei um ano inteiramente às disciplinas. Estranhamente, fechei todas com B. No mestrado, fazia mais de 30 créditos na graduação, 4 disciplinas na pós, pesquisa, e fechei todas com A.
\item A literatura de para o trabalho proposto é ampla e este aprofundamento tem sido constante,
\item Os acréscimos aos modelos atuais tem tido o foco no participante da rede.
\item Há implementação computacional de provas de conceito, bibliotecas, rotinas básicas e rotinas para replicar resultados do grupo de pesquisa.

\item A escrita da tese pode tomar vários rumos: pode consistir de um conjunto de artigos ou de uma monografia final. Acho mais provável que seja um conjunto de artigos focados nas direções dadas na Seção~\ref{sec:com}.
\item Conseguimos finalizar um artigo~\cite{timeS}. Há ao menos mais um em condições de publicação~\cite{ops} e outro mais indiretamente relacionado sobre música~\cite{mass}. Além destes, há mais estes artigos no arXiv~\cite{ensaio,connectiveDiff,XXX,YY}, todos referentes ao trabalho do doutorado. Foram publicados em revista internacional os artigos AA e Images/Vilson, ambos sem a colaboração do orientador.
\item Parte substancial do trabalho consistiu em experimentos de coleta e difusão de informação, o que disparou reuniões, visitas e colaborações. Este processo foi iniciado logo antes do doutorado e pode ser apreciado, por exemplo, pelas visitas a São Carlos de parcerios de pesquisa, pela integração do pesquisador ao grupo de pesquisa Nexus, vinculado ao CNPq, e ao aporte do PNUD/ONU dado ao pesquisador, sobre o qual a Presidência da República se posicionou como beneficiária.
\end{enumerate}

\chapter{Conclusões e previsão}
Há, a princípio, uma confirmação de que os conhecimentos de redes complexas
possuem aplicações diversas e potencialmente benéficas para o participante.
Por exemplo, os experimentos apresentaram modificações da estrutura social para comportar a pesquisa, e podem ser usados para comportar outros empreendimentos. Os estudos de estabilidade e diferenciação em redes de interação humana apontam na direção de tipologias de redes e de participantes, com base nos setores de Erdös e com componentes principais típicas e estáveis.
Um legado de dados ligados e abertos é conveniente para apresentar estes resultados às comunidades acadêmicas e interessadas nas aplicações, para as quais foram adiantadas ontologias, vocabulários, rotinas de conversão de dados relacionais em RDF e os dados em si.

Uma direção simples
é focar no \emph{Complex Networks Gradus ad Parnassum},
que está planejado como uma apresentação das redes complexas através
da entrega, para o leitor,de formas de observar e interagir com suas redes.
Uma via menos pedagógica, mas mais fundamental, é
explorar as estabilidades encontradas: até que número de
agentes a distribuição dos participantes nos setores e a formação das componentes principais se mantém?
Como caracterizar a intermitência dos agentes enquanto a distribuição de
grau é estável?

Há, em alguns casos extremos, considerações na base da área,
com implicações sobre a própria constituição das redes complexas.
Ao mesmo tempo, os métodos utilizados são potencialmente novos.
Há diversos trabalhos na bibliografia.
Caso haja disponibilidade para visitar itens da literatura
produzida, recomendamos, nesta mesma ordem,~\cite{timeS,pnud5,ensaioAA,gmane,fourHubs}.

\bibliography{biblio}

\end{document}

