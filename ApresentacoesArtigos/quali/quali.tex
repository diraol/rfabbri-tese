\documentclass[a4paper,openright,12pt]{report} %,parskip=full

\usepackage{amssymb,amsmath,textcomp}
\usepackage[brazil,english]{babel}
\usepackage[utf8]{inputenc}
\usepackage{braket}
\usepackage{booktabs}  %Possibilidade de linhas mais grossas nas tabelas.

%MEU
%\usepackage[brazilian, portuguese, activeacute]{babel}
%\usepackage{graphicx}
\usepackage{wrapfig}
\usepackage{subfigure}
%\usepackage{color}
%\usepackage{amssymb}
\usepackage{amsmath}  %for binom, not
%\usepackage{pifont}   %for ding
\usepackage{hyperref}
%\usepackage{amsthm}   %for what
%\usepackage{helvet}   %for what
\usepackage{cancel}   %for cancel
\hypersetup{
    colorlinks,
    citecolor=black,
    filecolor=black,
    linkcolor=black,
    urlcolor=black
}

\usepackage{etex}

\usepackage{cite}



%\usepackage[footnotesize,hang]{caption} 

%\usepackage[hang,small,labelsep=endash]{caption} % Hifen na legenda de tableas e figuras
\usepackage[hang,footnotesize,labelsep=endash,tableposition=top]{caption} % Hifen na legenda de tableas e figuras
%CHCHCH
%\usepackage{subcaption}

%%%%%%%%%%%%%%%%% Bibliografia
\usepackage{url}
\usepackage[bibjustif,abnt-thesis-year=both,num]{abntcite} %%%%%% ABNY
%%%%%%%%%%%%%%%%% Espacamento
%\makeatletter  %Espa�amento entre os itens da bibliografia
%\newcommand{\adjustmybblparameters}{\setlength{\itemsep}{2\baselineskip}\setlength{\parsep}{0.5ex}}
%\let\ORIGINALlatex@openbib@code=\@openbib@code  
%\renewcommand{\@openbib@code}{\ORIGINALlatex@openbib@code\adjustmybblparameters}
%\makeatother
%%%%%%%%%%%%%%%%%%%%%%%%%%%%%%%%%%%
% Espacaçamento entre as referências
\makeatletter  %Espa�amento entre os itens da bibliografia
\newcommand{\adjustmybblparameters}{\setlength{\itemsep}{2\baselineskip}\setlength{\parsep}{0.5ex}}
\let\ORIGINALlatex@openbib@code=\@openbib@code  
\renewcommand{\@openbib@code}{\ORIGINALlatex@openbib@code\adjustmybblparameters}
\makeatother

%%%%%%%%%%%%%%%%%%%%%%%%%%%%%%%%%%%%%%%%%%%%%%%
%%%%%%%%%%%%%%%%%%%%%%%%%%%Symbol footnote
\long\def\symbolfootnote[#1]#2{\begingroup%
\def\thefootnote{\fnsymbol{footnote}}\footnote[#1]{#2}\endgroup}
%%%%%%%%%%%%%%%%%%%%%%%%%%

\usepackage[Conny]{fncychap}
\usepackage{appendix}

%%%%%%%%%%%%%%%%%%%%%%%%%%%%%%%%%%%%%%%5

\makeatletter
\renewcommand*{\@makechapterhead}[1]{%
  \vspace*{-1.0cm}%
  {\parindent \z@ \raggedright \normalfont
    \ifnum \c@secnumdepth >\m@ne
      \if@mainmatter%%%%% Fix for frontmatter, mainmatter, and backmatter 040920
        \DOCH
      \fi
    \fi
    \interlinepenalty\@M
    \if@mainmatter%%%%% Fix for frontmatter, mainmatter, and backmatter 060424
      \DOTI{#1}%
    \else%
      \DOTIS{#1}%
    \fi
  }}

%older
%%%%%%%% Espaçamento entre títulos de seções e linhas anteriores e subsequentes
%\usepackage[compact]{titlesec}
%\titlespacing{\section}{0pt}{0.5cm}{0.5cm}
%%\titlespacing{\chapter}{0pt}{2.0cm}{2.0cm}
%\titlespacing{\subsection}{0pt}{0.5cm}{0.5cm}
%\titlespacing{\subsubsection}{0pt}{0.5cm}{0.5cm}	

%%abnt mestrado
%%%%%%%%% Espaçamento entre títulos de seções e linhas anteriores e subsequentes
%\usepackage{titlesec} %[compact]
%\titlespacing{\chapter}{0pt}{1.9cm}{2.2cm}
%\titlespacing{\section}{0pt}{1.9cm}{2.2cm}
%\titlespacing{\subsection}{0pt}{2.1cm}{2.2cm}
%\titlespacing{\subsubsection}{0pt}{2.2cm}{2.2cm}	

%quali
%%%%%%%% Espaçamento entre títulos de seções e linhas anteriores e subsequentes
\usepackage{titlesec} %[compact]
\titlespacing{\chapter}{0pt}{0cm}{1.5cm}
\titlespacing{\section}{0pt}{1.5cm}{1.5cm}
\titlespacing{\subsection}{0pt}{1.5cm}{1.5cm}
\titlespacing{\subsubsection}{0pt}{1.5cm}{1.5cm}	

%  \usepackage{titlesec}
  \titleformat{\chapter}[display]
  {\normalfont\bfseries\filcenter}
  {\LARGE\thechapter}
  {1ex}
  {\titlerule[2pt]
	  \vspace{2ex}%
  \LARGE}
  [\vspace{1ex}%
  {\titlerule[2pt]}]


%%%%%%%%%% Escreve ``Figura`` e ``Tabela`` nas listas de figuras e tabelas
%%CHCHCH
%\usepackage[titles]{tocloft}
%\renewcommand{\cftfigaftersnum}{ -}
%\renewcommand{\cfttabaftersnum}{ -}
%\renewcommand{\cftfigpresnum}{Figura }
%\renewcommand{\cfttabpresnum}{Tabela }
%\setlength{\cftfignumwidth}{2.2cm}
%\setlength{\cfttabnumwidth}{2.2cm}
%%%%%%%%%%%%%%%%%%%%%%%%%%%%%%%%


\usepackage[pdftex]{graphicx}  %pacote para colocar figuras jpg
\usepackage[final]{pdfpages}  %include pfd coloca um pdf em uma pagina

%\usepackage[T1]{fontenc} %CHCHCH
%\usepackage[OT2,T1]{fontenc} %CHCHCH
%\usepackage[T1,T2A]{fontenc} %needs t2aenc.def
\usepackage[T1,OT2,OT1]{fontenc} %CHCHCH

%\usepackage{cyrillic} 

%russian
\newcommand\cyr{%
\renewcommand\rmdefault{wncyr}%
\renewcommand\sfdefault{wncyss}%
\renewcommand\encodingdefault{OT2}%
\normalfont
\selectfont}
\DeclareTextFontCommand{\textcyr}{\cyr}

%%CHCHCH
%%russian
%\newcommand{\cyrrm}{\fontencoding{OT2}\selectfont\textcyrup}
%\newcommand{\cyrit}{\fontencoding{OT2}\selectfont\textcyrit}
%\newcommand{\cyrsl}{\fontencoding{OT2}\selectfont\textcyrsl}
%\newcommand{\cyrsf}{\fontencoding{OT2}\selectfont\textcyrsf}
%\newcommand{\cyrbf}{\fontencoding{OT2}\selectfont\textcyrbf}
%\newcommand{\cyrsc}{\fontencoding{OT2}\selectfont\textcyrsc}
%%%%% cyrrm = "Roman", or really upright, normal font
%%%%% cyrit = Italic (cursive forms of letters)
%%%%% cyrsl = Italic (non-cursive forms of letters)
%%%%% cyrsf = Sans-serif
%%%%% cyrbf = Bold-face 




%\usepackage[math]{iwona}
\renewcommand{\familydefault}{\sfdefault}
%\renewcommand{\familydefault}{\sfdefault} math

\usepackage[absolute]{textpos}
\usepackage[retainorgcmds]{IEEEtrantools}
\usepackage{leftidx}

%CHCHCH
%\usepackage[superscript]{cite}  %citacoes em superscript
\usepackage{cite}  %nao, meu, citacoes em superscript nao!

\usepackage{tikz}
\usetikzlibrary{arrows}
\usetikzlibrary{decorations}
\usetikzlibrary{snakes}

\usepackage{color}

\definecolor{DarkBlue}{rgb}{0.1,0.1,0.5}
\definecolor{Red}{rgb}{0.9,0.0,0.1}
\definecolor{DarkGreen}{rgb}{0.10,0.50,0.10}

% see documentation for a0poster class for the size options here
\let\Textsize\large
\def\Head#1{\noindent\hbox to \hsize{\hfil{\LARGE\color{DarkBlue} #1}}\bigskip}
\def\LHead#1{\noindent{\LARGE\color{DarkBlue} #1}\smallskip}
\def\Aut#1{\noindent{\Huge\color{DarkBlue} #1}\smallskip}
\def\End#1{\noindent{\large\it\color{DarkBlue} #1}\smallskip}
\def\Subhead#1{\noindent{\Large\color{DarkBlue} #1}}
\newcommand{\quiteHuge}{\fontsize{120}{93}\selectfont}
\def\Title#1{\begin{center}\noindent{\quiteHuge\color{DarkGreen}#1}\end{center}}


\usepackage[twoside,inner=3cm,outer=2cm,top=3cm,bottom=2cm]{geometry}

\usepackage{setspace}   %Espacamento entre as linhas

\setstretch{1.5} %Normal da ABNT: espaçamento entre as linhas de 1.5 de linha
% Pode ser \singlespace 
%          \onehalfspace
%          \doublespace

%---------------------------------------------------------------------------------------------------------
\usepackage{fancyhdr}
\pagestyle{fancy} % colocar Capítulos, Seções, etc em minúsculo

\renewcommand{\sectionmark}[1]{\markright{\thesection\ #1}}

\fancyhf{} % deletar configuração atual do cabeçalho (header) e rodapé (foot)

\pagestyle{fancy} 

\fancyhead[LE,RO]{\thepage}

\fancyhead[LO]{\rightmark}
  
\fancyhead[RE]{\leftmark}

\renewcommand{\headrulewidth}{0.5pt}
  
\renewcommand{\footrulewidth}{0pt}

%CHCHCH
%\addtolength{\headheight}{0.5pt} % cria um espaço para linha
\addtolength{\headheight}{4.0pt} % cria um espaço para linha

\fancypagestyle{plain}{
    \fancyhead{} % exibir cabeçalho e rodapé
    \renewcommand{\headrulewidth}{0pt} % linha
}




%---------------------------------------------------------------------------------------------------------

\usepackage[marginal,symbol]{footmisc}
\footnotemargin2pt

%%%%%%%%%%%%%%%%%%


\pagestyle{empty}

%\AtBeginDocument{\addtocontents{toc}{\protect\thispagestyle{empty}}} %CHCHCH %to make tableofcontents empty pagestyle


%CHCHCH TO MAKE TABLEOFCONTENTS PAGESTULE EMPTY
%\fancypagestyle{plain}{%
%  \fancyhf{}                          % clear all header and footer fields
%  \renewcommand{\headrulewidth}{0pt}
%  \renewcommand{\footrulewidth}{0pt}
%}

\usepackage[subfigure,titles]{tocloft}
\renewcommand{\cftfigaftersnum}{ -}
\renewcommand{\cfttabaftersnum}{ -}
\renewcommand{\cftfigpresnum}{Figura }
\setlength{\cftfignumwidth}{2.2cm}%{5em}



%\renewcommand*{\chapterheadendvskip}{%
%  \vspace{3.0cm}%
%%  \vspace{0.725\baselineskip plus 0.115\baselineskip minus 0.192\baselineskip}%
%}

\usepackage{blindtext}

%%%%%%%%%%%%%%%%%%%%%%%%%%%%%%%%%%%%%%%%%%%%%%%%%%%%%%%%%%%%%%%%%%%%%%%%%%%%%%%%%%%%%%%%%%%%%%%%%%%%%%%%%%%%%%%%%%%%%%%%%%%%%%%%%%%%%%%%%%%%%%%%%%%
%%%%%%%%%%%%%%%%%%%%%%%%%%%%%%%%%%%%%%%%%%%%%%%%%%%%%%%%%%%%%%%%%%%%%%%%%%%%%%%%%%%%%%%%%%%%%%%%%%%%%%%%%%%%%%%%%%%%%%%%%%%%%%%%%%%%%%%%%%%%%%%%%%%

\begin{document}

%\setlength{\parskip}{0.5cm}%1.5ex} %0pt} % 1ex plus 0.5ex minus 0.2ex}

\hyphenation{ca-ra-te-ri-sti-cas}

%\selectlanguage{portuguese}
%\hyphenation{e-mer-g\^{e}n-cia}


\pagestyle{fancy}


\thispagestyle{empty}

\vspace{0.5cm}

\begin{center} 
\LARGE{UNIVERSITY OF SÃO PAULO}  \\
\LARGE{SÃO CARLOS PHYSICS INSTITUTE}
\end{center}

\vspace{6.0cm}

\centerline{\LARGE{RENATO FABBRI}}

\vspace{3.0cm}


\centerline{\Huge{Complex networks for the participant}} 
\vspace{0.5cm}
%\centerline{\Huge{aplicada a reconhecimento de autoria}}


\vspace{6.5cm}

\begin{center}
\Large{S\~ao Carlos}\\
\Large{2015}
\end{center}



\newpage\ \thispagestyle{empty}  \newpage\thispagestyle{empty}

\setcounter{page}{1} % ABNT: deve-se começar a numeração na folha de rosto.

\begin{center}
\LARGE{RENATO FABBRI}
\end{center}

\addvspace{4.0cm}

\begin{center}
\Huge{Complex networks for the participant}
\end{center}

\addvspace{3.0cm}

\makebox[15cm][r]{
\begin{minipage}[l]{8cm}

\begin{singlespace}
Monograph presented to the Physics Graduate Program of the
São Carlos Institute of Physics, University of São Paulo,
for the qualification exam, as part of the requirements
for obtaining the title of Doctor in Sciences.\\

Concentration area: Applied Physics

Option: Computational Physics

Advisor: Prof. Dr. Osvaldo Novais de Oliveira Jr.
\end{singlespace}


\end{minipage}}

\addvspace{3.0cm}

\begin{center}
%\Large{Versão original} \\
\vspace{1.5cm}
\Large{S\~ao Carlos}\\
\Large{2015}
\end{center}


%%%%%%
%\thispagestyle{empty}
%
%
%\includepdf{index}
%\includepdf{fichacatalog135}
%%%%%%

%% \vspace*{8cm}
%%\begin{center}
%% \huge{FICHA CATALOGRÁFICA}

%% \vspace*{3cm}

%%\large{Elaborar ao final, quando o número de páginas da 
%%dissertação estiver definido - 
%%www.biblioteca.ifsc.usp.br/ficha}
%%\end{center}

%%%%%%%%%%%
%\newpage\thispagestyle{empty} 
%
%\vspace*{8cm}
%
%\begin{center}
%\normalsize{FOLHA DE APROVAÇÃO}
%\end{center}
%
%
%
%\newpage\ \thispagestyle{empty}  \newpage\thispagestyle{empty}\ 
%
%\vspace*{18cm}
%
%\begin{spacing}{1.2}
%
%\begin{flushright}\textit{
%\`A MINHA AV\'O,\\
%\vspace{0.5cm}
%CLELIA C\'ESPEDES ACERO
%\vspace{1cm}
%}\end{flushright}
%
%\end{spacing}
%%%%%%%%%%%



%%%%%%%%%%%%%%%%%%%%
%\newpage\ \thispagestyle{empty} %NEWW
%
%\newpage\ \thispagestyle{empty}
%
%
%\vspace*{18cm}
%
%\begin{spacing}{1.2}
%
%\makebox[15cm][r]{ 
%\begin{minipage}[l]{10cm}
%\hspace*{2.2cm}\emph{ \large{ Este trabalho foi financiado pelo \\ \hspace*{7.5cm} CNPq } }
%\end{minipage}}
%
%
%
%\end{spacing}
%%%%%%%%%%%%%%%%%%%%





%\newpage\ \thispagestyle{empty}  \newpage\thispagestyle{empty}

\newpage\ \thispagestyle{empty}  \newpage\thispagestyle{empty}


\begin{singlespace}

\centerline{\LARGE{{\bf ABSTRACT}}}

\vspace*{1.5cm}

\hspace*{-0.9cm} FABBRI, C. \textit{Complex networks for the participant}. 
São Carlos Institute of Physics. University of São Paulo, São Carlos, 2015. 

\vspace*{1.2cm}

\hspace*{-0.9cm}

{\noindent
	Complex networks form one of the most active fields of recent physics.
	With respectable efforts for exhibiting advances to the general audience, 
	it seems, however, that few or none of these are targeted to the individual profit.
	That is, with a core knowledge about the field, and recipes for harnessing,
	provide means for the participant to interact and understand the networks they are in.
	This work aims to accomplish such task by means of the social networks of the participants.
	We verified basic ubiquitous characteristics in such networks, such as time stability of topological measures,
	of basic connective sector sizes and the differentiation of the textual production.
	We also formalized conceptualizations of these networks as OWL were they were possible, specially
	in relation to the social participation instances provided by law. Finally, software and
	data have been put available and used, as means to enable integrated analysis of different provenance and public benefit.
	Conceptual consequences have been documented and requires anthropological considerations.
	Furthermore, software, ontological and data contributions can be better documented and developed while
	a typological consideration of the physical properties observed in human interaction networks should
	bridge complex networks and more the traditional legacy of human sciences on the subject.
}

\vspace*{1.5cm}
\hspace*{-0.9cm} {\bf Keywords:} Complex networks. Social networks. Complexity. Anthropological physics. Linked data. Semantic web. Social participation. Text mining. Natural language processing.

\end{singlespace}

  

%%%%%%%%%%%%%%%%%%%%%%%%%%%%%
%\newpage\thispagestyle{empty}
%
%\pagestyle{empty}
%
%\begin{singlespace}
%\listoffigures
%%\listoftables
%\end{singlespace}
%%%%%%%%%%%%%%%%%%%%%%%%%%%%%

%%%
%\newpage\ \thispagestyle{empty}  \newpage\thispagestyle{empty}

\newpage\ \thispagestyle{empty}  \newpage\thispagestyle{empty}
%\thispagestyle{empty} %CHCHCH

\tableofcontents\thispagestyle{empty}\thispagestyle{empty}%\thispagestyle{empty}

\clearpage \thispagestyle{empty}

\pagestyle{fancy}

\chapter{Introduction}
Studies on human interaction networks have started long before modern computers, dating back to the nineteenth century, while the foundation of
social network analysis is generally attributed to the psychiatrist Jacob Moreno in mid twentieth century~\cite{newmanBook}.
With the increasing availability of data related to human interactions, research on these networks has grown continuously.
Contributions can now be found in a variety of fields, from social sciences and humanities~\cite{latour2013} to computer science~\cite{bird} and physics~\cite{barabasiHumanDyn,newmanFriendship}, given the multidisciplinary nature of the topic.
One of the approaches from an exact science perspective is to represent interaction networks as complex networks~\cite{barabasiHumanDyn,newmanFriendship}, with which 
several features of human interaction have been revealed.
For example, the topology of human interaction networks exhibits a scale-free trace, which points to the existence of a small number of highly connected hubs and a large number of poorly connected nodes.

There is gap of knowledge and technology between the complex networks legacy and the usufruct of the participant. This hiatus is reactive, and events suggest that it will maintain itself as an ecosystem of knowledge, technology and undertake of society in all scales. It should ease for example: entrepreneur goals, elaboration and preparation of documents, rapid achievement of knowledge. In general: processes that refer to collection and diffusion of goods and information.
% Argumentar que é tido que o correto entendimento destas estruturas
% será mais forte que XXX (ver no curso da Melanie/CE).

This work presets a confirmation of such scenario and advances.
Some strategies were selected to verify the complex networks to the
advantage of the participant. Specially, very simple experiments
seems able to modify social structures. In this context, we verified
temporal stabilities in human interaction networks, and exposed that
primitive sectors of the networks (hubs, intermediary and periphery)
produce texts that are very different. This is useful for a
not stigmatizing typology of participants in interaction networks
via quantitative criteria. This yielded audiovisualization and
interconnection of data as art and technological gadgets, 
as support for scientific research. Applications were complemented
with the (Brazilian) Presidency of the Republic and UNDP.

Next section presents directions on related work. Section~\ref{sec:mat} is dedicated to the data analyzed. Section~\ref{sec:met} holds the methods used to reach results. Results are given in Section~\ref{sec:res}. The chronogram and outline of finished, ongoing and planed tasks are in Section~\ref{sec:chr}. Conclusions are in Section~\ref{sec:con} followed by acknowledgments and the bibliography.
% Interacting with own network, with constant feedback, study of the self
% Annotation of own data, study of them. Extend analysis within the route.
%In this work, we present advances in handing the participant ways to explore and harness the social networks he/she participates in.
%This is not an easy task, therefore we selected

\section{Literature review}
The field  of complex networks is relatively new ($\approx 25$ years)
and literature presents diverging definitions.
One definition that is having increasing acceptance
considers a complex network a ``large graph with non-trivial topological features''.
This definition is misleading in at least two points: there are networks of interest with trivial topological features, such as the paradigmatic Erdös-Rényi network and the lattice network~\cite{newman}. 
Second, it fails to deliver the fundamental message that the complex network is not an isolated mathematical graph structure. Complex networks of interest are real networks or idealized models for understanding them. Besides that, not only large graphs are of interest, but small graphs are very often used as toy examples and measured as extension of larger structures.
A definition, still far from perfect, but preferred in this work, is:
%``often large graphs considered in the physical
%or natural environment they reside''.
``often large and non-trivial graphs considered in,
or for the consideration of, 
the environment they reside''.
This definition resolves both issues.

Books in general present a common and powerful repertory for characterization of complex systems through graphs. Maybe most importantly are the arsenal of measures: degree, strength, betweenness centrality, clustering coefficient, etc.; the basic network paradigms: Erd\"os Rényi, geographical, small-world, scale-free; and the transdisciplinary approach to the environments that yield the networks. The literature on social network analysis,foe example, can often be understood as dealing with complex networks in human social systems.

A careful consideration of the books and articles read for this research
is given in Section~\ref{}.

\chapter{Materials}
\section{The Gmane public database of email lists (benchmarks)}
\section{Facebook, Twitter, Participa.br, Cidade Democrática, AA}
\section{My own social networks}
Considerations about the right to annotate.

\chapter{Methods}
\section{Circular statistics}
\section{Erdös Sectioning}
\section{PCA of measures along time}
\section{Kolmogorov-Smirnoff test for texts produced by sectors}
\section{Audiovisualization of data}
\section{Typological considerations}
\section{Semantic web}
\subsection{OWL ontology construction}

\chapter{Results}
\section{Time stability in human interaction networks}
\section{Semantic web}
\subsection{Linked data}
\subsection{RDF data conversion of data into linked data}
\subsection{Published linked data and OWL ontologies}
\section{Harnessing}
\subsection{Social percolation procedures}
\subsection{Recommendation systems for the enrichment of semantic navigation}
\subsection{Understanding the social being}
Scale free as the consequence of $T^2$ signal, fractal, constant, with three primitive parts and greater specialization. Gradus ad Parnassum

\chapter{Finished and planed tasks, chronogram}
% colocar comparativos aqui
\section{Documents}
\subsection{To be finished}
\subsubsection{Anthropological physics}
The study of human systems raises conceptual
and ethical issues that require anthropological considerations.
There are two immediate routes to this concepts:
\begin{itemize}
	\item What data should or can be used?
	\item Can one experiment in a network of humans? In which context?
\end{itemize}

The short answer is that ethics committees and procedures are dedicated to dealing with those issues.
Even so, there is a key-concept from the anthropological legacy: the study of the self as exposed to the interested culture or context. 
In this sense, it is reasonable (if not a suggestion) that
a researcher do reflexive consideration, 
i.e. that he/she observe and make assumptions about its own sampling of the world.
Within this same framework, many social networks (email, Facebook, Twitter, Participa.br, AA) were openly mined,
with feedback to and from the studied communities.
The term ``anthropological physics'' started being used in Brazil around
2014 and can be thought as a subfield of Social Physics.

\subsubsection{Gradus}
% 1/N^2 == physics of humans, they are the agents
Fazer o $2^(100 2)$

Consider a idealized constitution of these networks: 
\begin{itemize}
	\item the resources of the environment are the persons, each with an amount of time available.
	\item The amount of resource employed by the environment to the network is constant through all connective sectors
\end{itemize}

\subsection{Finished}


\section{Chronogram}

\begin{center}
\begin{table}[ht]
\centering
\begin{tabular}{lll}
Ano & Semestre & Atividade \\
%\hline
\hline
2012 & II & Revis\~ao e estudo da bibliografia.\\
\hline
2013 & I & Implementa\c c\~ao computacional.\\
%\hline
\ & II & Cursar disciplinas.\\
\hline
2014 & I & Implementa\c c\~ao computacional: refinamento do c\'odigo e corpus.\\
%\hline
\ & II & Apresenta\c c\~ao de resultados.\\
\hline
2015 & I & Exame de qualifica\c c\~ao.\\
\ & \ & Escrita de artigo.\\
%\hline
\ & II & Cursar disciplina.\\
%\ & \ & Monitoria para o Programa de Aperfei\c c\~ao do Ensino.\\
\ & \ & Monitoria PAE.\\
\hline
2016 & I & Defesa do doutorado.\\
\ & \ & Escrita de artigo.\\
\hline
\end{tabular}
\caption{Cronograma de atividades}
\label{table:cronograma}
\end{table}
\end{center}



\chapter{Conclusions}

\bibliography{biblio}

\end{document}

