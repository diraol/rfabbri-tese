% Suggested LaTeX template for a UC Davis Mathematics Qualifying Exam
% Proposal

% For official requirements, you should consult directly with your
% advisor in addition to the Brochure specific to your graduate program.

\documentclass{article}

    \usepackage{amsfonts} % For AMS Beautification
    \usepackage{fullpage} % For More Realistic Page Usage

\begin{document}
  
    \title{\bf Stability and characterization of human interaction networks}
    
    % Insert your name below
    \author{Renato Fabbri}

    % Insert a specific date below if appropriate
    \date{Draft Date: \today}
    
    \maketitle
 
    
    %%%%%%%%%%%%%%%%%%%%%%%%%%%%%%%%%%%%%%%%%
    %                                       %
    %       Exam Meta Information Section   %
    %                                       %
    %%%%%%%%%%%%%%%%%%%%%%%%%%%%%%%%%%%%%%%%%
    \begin{tabular}{c @{ }c}
        %
        % Exam Committee Subsection
        %
        \begin{minipage}[t]{0.5\linewidth}
            \section*{Exam Committee}
            \begin{description}
                
                \item[Committee Chairperson:] \begin{tabular}{c} \\ \end{tabular}
                
                    Prof. [Insert Chairperson's Name]
                   
                \item[Committee Members:] \begin{tabular}{c} \\ \end{tabular}
                
                    Prof. [Insert Committee Member's Name]
                
                    Prof. [Insert Committee Member's Name]
                
                    Prof. [Insert Committee Member's Name]
                
                    Prof. [Insert Committee Member's Name]
                    
            \end{description}
        \end{minipage}  
        &
        %
        % Exam Logistics Subsection
        % 
        \begin{minipage}[t]{0.5\linewidth}
            \textbf{\Large Exam Logistics}
            \begin{description}
                
                \item[Date:] \begin{tabular}{c} \\ \end{tabular}
                
                    [Insert Exam Date with Day of the Week]
                            
                \item[Time:] \begin{tabular}{c} \\ \end{tabular}
                
                    [Insert Exam Start Time]
                    
                \item[Location:] \begin{tabular}{c} \\ \end{tabular}
                
                    [Insert Exam Building and Room Number]
                    
            \end{description}
        \end{minipage}
    \end{tabular}
    \smallskip
    
    
    %%%%%%%%%%%%%%%%%%%%%%%%%%%%%%%%%
    %                               %
    %       Research Talk Section   %
    %                               %
    %%%%%%%%%%%%%%%%%%%%%%%%%%%%%%%%%
    \section*{Proposed Research Talk}
        \begin{description}
            \item[Title.]  Use and understandings of human interaction networks
        
            \item[Abstract.]  Human interaction networks from arts to social participation. Government art, civil art, activism, real time. Primitive typology, topological stability, textual differentiation. Anthropological physics experiments. Results: gmane toolkit, knowledge, social tecnologies like social structures streaming, ontologies, etc.
                        
        \end{description}
                
        
    %%%%%%%%%%%%%%%%%%%%%%%%%%%%%%%%%
    %                               %
    %       Exam Syllabus Section   %
    %                               %
    %%%%%%%%%%%%%%%%%%%%%%%%%%%%%%%%%
    \section{Introduction}
    Ubiquity of human networks since mankind origins.
    Digital traces eases uses and understandings.
    Has been so important that we have a revolution.
    Government art, civil art (MMISSA, MyNSA, Telões, Versinus)
    \subsection{Practical and cientific goals}
    \subsection{Complementary routes and linked data}
    Ontologies and RDF representation of data delivers knowledge representation
    proper for civil uses for the following reasons:
    \begin{itemize}
	\item Integration of different instances.
	\item Semantics are integrated to data.
	\item Formal representations for automated reasoning.
	\item Formal representations of conceptualizations, easing qualitative approaches to data.
	\item Standards in use and handled by the W3C.
    \end{itemize}

    These are not the focus of the work, but have been delivered by the candidate~\cite{OPS,OPa,OCD,Ontologiaa,OBS,VBS}.

    Other possibilites envisioned are more linked to traditional approaches to politics and humanities, which were not the focus of the endeavors, only referred to in the delivered articles~\cite{stability,differentiation,versinus,ops}.

    \section{Materials}
    Gmane database, facebook twitter.
    \section{Methods}
    Distribution of activity along time and among participants.
    PCA. Additional measures of symmetry, comparrisson with erdos renyi network. kolmogorov-smirnov, visualization, exposition of the researcher to the networks, exposition of results to the networks.
    \section{Results}
    \subsection{Stability and differentiation of a primitive typology}
    \subsection{Technologies}
    \subsubsection{Ontologies}
    \subsubsection{Social streaming}
    \subsubsection{Gmane Toolkit}
    \subsection{Art}
    \section{Conclusions}

%        \begin{description}
%            \item[Topic \#$1$:] Complex networks
%            
%                \begin{itemize}
%            
%                    \item Interaction networks \\
%                    
%                    Reference: Pages or Chapters from a well-known
%                    reference.
%                            
%                    \item Human networks                    \\
%                    
%                    Reference: another well-known
%                    reference.		    
%		    
%                \end{itemize}
%                
%        
%            \item[Topic \#$2$:]         
%            \item[$\hspace{0.75cm}\vdots$]
%        
%            \item[Topic \#$n$:] [Insert Specialization Within This 
%            Topic (with specific subitems and references as above)]
%        
%        \end{description}
    
  
\end{document}
