%%%%%%%%%%%%%%%%%%%%%%%%%%%%%%%%%%%%%%%%%
% Proceedings of the National Academy of Sciences (PNAS)
% LaTeX Template
% Version 1.0 (19/5/13)
%
% This template has been downloaded from:
% http://www.LaTeXTemplates.com
%
% Original author:
% The PNAStwo class was created and is owned by PNAS:
% http://www.pnas.org/site/authors/LaTex.xhtml
% This template has been modified from the blank PNAS template to include
% examples of how to insert content and drastically change commenting. The
% structural integrity is maintained as in the original blank template.
%
% Original header:
%% PNAStmpl.tex
%% Template file to use for PNAS articles prepared in LaTeX
%% Version: Apr 14, 2008
%
%%%%%%%%%%%%%%%%%%%%%%%%%%%%%%%%%%%%%%%%%

%----------------------------------------------------------------------------------------
%	PACKAGES AND OTHER DOCUMENT CONFIGURATIONS
%----------------------------------------------------------------------------------------

%------------------------------------------------
% BASIC CLASS FILE
%------------------------------------------------

%% PNAStwo for two column articles is called by default.
%% Uncomment PNASone for single column articles. One column class
%% and style files are available upon request from pnas@nas.edu.

%\documentclass{pnasone}
\documentclass{pnastwo}

%------------------------------------------------
% POSITION OF TEXT
%------------------------------------------------

%% Changing position of text on physical page:
%% Since not all printers position
%% the printed page in the same place on the physical page,
%% you can change the position yourself here, if you need to:

% \advance\voffset -.5in % Minus dimension will raise the printed page on the 
                         %  physical page; positive dimension will lower it.

%% You may set the dimension to the size that you need.

%------------------------------------------------
% GRAPHICS STYLE FILE
%------------------------------------------------

%% Requires graphics style file (graphicx.sty), used for inserting
%% .eps/image files into LaTeX articles.
%% Note that inclusion of .eps files is for your reference only;
%% when submitting to PNAS please submit figures separately.

%% Type into the square brackets the name of the driver program 
%% that you are using. If you don't know, try dvips, which is the
%% most common PC driver, or textures for the Mac. These are the options:

% [dvips], [xdvi], [dvipdf], [dvipdfm], [dvipdfmx], [pdftex], [dvipsone],
% [dviwindo], [emtex], [dviwin], [pctexps], [pctexwin], [pctexhp], [pctex32],
% [truetex], [tcidvi], [vtex], [oztex], [textures], [xetex]

\usepackage{graphicx}

\usepackage{hyperref}
%------------------------------------------------
% OPTIONAL POSTSCRIPT FONT FILES
%------------------------------------------------

%% PostScript font files: You may need to edit the PNASoneF.sty
%% or PNAStwoF.sty file to make the font names match those on your system. 
%% Alternatively, you can leave the font style file commands commented out
%% and typeset your article using the default Computer Modern 
%% fonts (recommended). If accepted, your article will be typeset
%% at PNAS using PostScript fonts.

% Choose PNASoneF for one column; PNAStwoF for two column:
%\usepackage{PNASoneF}
%\usepackage{PNAStwoF}

%------------------------------------------------
% ADDITIONAL OPTIONAL STYLE FILES
%------------------------------------------------

%% The AMS math files are commonly used to gain access to useful features
%% like extended math fonts and math commands.

\usepackage{amssymb,amsfonts,amsmath}

%------------------------------------------------
% OPTIONAL MACRO FILES
%------------------------------------------------

%% Insert self-defined macros here.
%% \newcommand definitions are recommended; \def definitions are supported

%\newcommand{\mfrac}[2]{\frac{\displaystyle #1}{\displaystyle #2}}
%\def\s{\sigma}

%------------------------------------------------
% DO NOT EDIT THIS SECTION
%------------------------------------------------

%% For PNAS Only:
%\contributor{Submitted to Proceedings of the National Academy of Sciences of the United States of America}
%\url{www.pnas.org/cgi/doi/10.1073/pnas.0709640104}
\copyrightyear{2015}
%\issuedate{Issue Date}
%\volume{Volume}
%\issuenumber{Issue Number}

%----------------------------------------------------------------------------------------

\begin{document}

%----------------------------------------------------------------------------------------
%	TITLE AND AUTHORS
%----------------------------------------------------------------------------------------

\title{Continuous voting by approval and participation} % For titles, only capitalize the first letter

%------------------------------------------------

%% Enter authors via the \author command.  
%% Use \affil to define affiliations.
%% (Leave no spaces between author name and \affil command)

%% Note that the \thanks{} command has been disabled in favor of
%% a generic, reserved space for PNAS publication footnotes.

%% \author{<author name>
%% \affil{<number>}{<Institution>}} One number for each institution.
%% The same number should be used for authors that
%% are affiliated with the same institution, after the first time
%% only the number is needed, ie, \affil{number}{text}, \affil{number}{}
%% Then, before last author ...
%% \and
%% \author{<author name>
%% \affil{<number>}{}}

%% For example, assuming Garcia and Sonnery are both affiliated with
%% Universidad de Murcia:
%% \author{Roberta Graff\affil{1}{University of Cambridge, Cambridge,
%% United Kingdom},
%% Javier de Ruiz Garcia\affil{2}{Universidad de Murcia, Bioquimica y Biologia
%% Molecular, Murcia, Spain}, \and Franklin Sonnery\affil{2}{}}

%\author{Renato Fabbri\affil{2}{University of S\~ao Paulo}
\author{Renato Fabbri\affil{2}{University of S\~ao Paulo}
\and
Ricardo Poppi\affil{1}{Brazilian Presidency of the Republic}
}
%James Smith\affil{2}{University of Oregon}
%\and
%Jane Smith\affil{1}{}
%}

\contributor{Draft}% to Proceedings of the National Academy of Sciences
%of the United States of America}

%----------------------------------------------------------------------------------------

\maketitle % The \maketitle command is necessary to build the title page

\begin{article}

%----------------------------------------------------------------------------------------
%	ABSTRACT, KEYWORDS AND ABBREVIATIONS
%----------------------------------------------------------------------------------------

\begin{abstract}
In finding the adequate way to prioritize proposals, the Brazilian participation community agreed about two indexes of approval and of participation.
Both practice and literature is constantly handled by the experts involved, and the formalization of such model and metrics seems novel.
Also, the relevance of this report is strengthened by the nearby use of these indexes by the Brazilian General Secretariat of the Republic to raise and prioritize proposals about public health care in open processes.
\end{abstract}

%------------------------------------------------

\keywords{social participation | recommendation systems | online voting | statistics} % When adding keywords, separate each term with a straight line: |

%------------------------------------------------

%% Optional for entering abbreviations, separate the abbreviation from
%% its definition with a comma, separate each pair with a semicolon:
%% for example:
%% \abbreviations{SAM, self-assembled monolayer; OTS,
%% octadecyltrichlorosilane}

% \abbreviations{}
%\abbreviations{PyPI, python package index}

%----------------------------------------------------------------------------------------
%	PUBLICATION CONTENT
%----------------------------------------------------------------------------------------

%% The first letter of the article should be drop cap: \dropcap{} e.g.,
%\dropcap{I}n this article we study the evolution of ''almost-sharp'' fronts

\section{}
%------------------------------------------------
% aqui já mando toda intro com as infos que cavei ontem.
Online decision making is a kind of recommendation system with special appeal for online social participation and electronic governments. This poses challenges on the design of such
processes regarding validity, security and the adequate indicators. Indeed,
the processes themselves vary, and the fact that the indexes presented here seem not to be
formalized and published is an evidence that such online decision making is very recent phenomena.

The main contribution of this report is a modeling for
an online voting process~\cite{issue1,issue2,tabRonald} with the following characteristics:
\begin{itemize}
    \item proposals might be inserted by voters after the voting phase started.
    \item Voting might be extended as a permanent process. In other words, voting on and adding new proposals might be open continuously.
    \item A proposal is presented to a voter one by one as random outcomes of all proposals.
    \item Each vote might be of one and only type among: ``approve'', ``disapprove'' and ``indifferent''.
    \item Voters vote without authentication.
    \item Intended mostly for national rankings, but can also be local or have foreign participation.
    \item Should result in a ranking of proposals to assist public management.
\end{itemize}

This setting requires care about security and validity.
Some of which are:
\begin{itemize}
	\item adequate sampling of individual proposals and overall ranking.
    \item Registration of the IP address and time of votes to ease detection of automated and other fraudulent efforts.
    \item Reasonable use of the outcomes from the voting process. This requires probing the survey being conducted and its purposes. The indexes here presented target indicatives for the Brazilian federal government about the most important health care proposals. Given the unauthenticated voting, the outcomes might be regarded as reference rankings if data is minimally shared and checked for inadequate data entry (such as voting by automated scripts or a persistent participant introduced bias).
\end{itemize}

% limiares
% PPA
% rastreando IP

\section{}

%\vfill
%\vspace{.6cm}
%\vfill
\subsection{Approval and participation indexes}
% equations of each participation index.

The approval index $\alpha_i$ and the participation index $\gamma_i$ of the proposal $i$ was defined as:
\begin{equation}
\begin{split}
\alpha_i=\frac{v^+_i-v^-_i}{\eta_i} \\
\gamma_i=\frac{v^+_i+v^-_i}{\eta_i}
\end{split}
\end{equation}
where $v^+_i$, $v^-_i$ and $\eta_i$ are approval count, disapproval count and exhibition count, respectively.
Note that $\alpha_i\in[-1,1]$, $\gamma_i\in[0,1]$, and $v^o_i=\eta_i-v^+_i-v^-_i$ is the count of the ``indifferent'' manifestations received by proposal $i$.
Also, such $\alpha_i$ and $\gamma_i$ indexes are expected, for each proposal $i$, to be a constant  plus a sampling estimate error that should be smaller as $\eta_i$ raises. This error is thought to be acceptable if $\eta_i$ is above a threshold $\overline{\eta}$ established by the participation community and public managers.
As an initial decision, the staff agreed to use $\overline{\eta}$ as to select $10-20\%$ of all proposals.
A threshold $\overline{\gamma}$ can be used as a required level of engagement for proposals to be relevant,
while the threshold $\overline{\alpha}$ is used to classify the outcome
as ``approved'',``disapproved'' and ``clash''. More specifically:

\begin{equation}
\begin{split}
\eta_i > \overline{\eta}    \;\;\;\;\;\; & \Rightarrow\;\;\;\;\;\; i     \text{ is sampled} \\
\gamma_i>\overline{\gamma}    \;\;\;\;\;\; & \Rightarrow\;\;\;\;\;\; i    \text{ is relevant} \\
|\alpha_i| \leq\overline{\alpha}\;\;\;\;\;\; & \Rightarrow\;\;\;\;\;\; i  \text{ is a clash} \\
\alpha_i > \overline{\alpha}  \;\;\;\;\;\; & \Rightarrow\;\;\;\;\;\; i    \text{ is approved} \\
-\alpha_i > \overline{\alpha} \;\;\;\;\;\; & \Rightarrow\;\;\;\;\;\; i   \text{ is disapproved} \\
\end{split}
\end{equation}

If a proposal is both sampled and relevant, than it is prioritized.
The coherent values of $\overline{\alpha_i}=0.5$ (or $1/3$)
and $\overline{\gamma_i}=0.5$ were chosen as standards of the decision model. 
These are likely to change with implementation and management. 


Thresholds might be dependent on proposal, such as given by coherent relations. Immediate examples are: $\overline{\gamma_i}=1-\frac{\eta_i}{max(\{\eta_j , \forall j \})}$ or $\overline{\alpha_i} = 1-\gamma_i$. These bonds among proposal variables has been discarded by the staff in present initial steps.







% notive that threasholds can be taken as different for each proposal: \overline{\alpha_i}=1-\eta_i/\eta_{max}

\vfill
\vspace{.6cm}
\vfill


\subsection{Selected decision framework examples}

Many of the online decision processes conceived and practiced resemble our model and have similar measurements to the $\alpha_i$ and $\gamma_i$ indexes. This section presents a collection of models more familiar to the Brazilian participatory community, with focus on the mechanisms, not on historical notes.

Pairwise~\cite{pairwise} is part of the tackled paradigm: the ranking procedure accepts new proposals while the voting occurs. Even so, pairwise voting is comparative, voter chooses between two proposals at each vote, and this does not fit proposed procedure.

Appgree software~\cite{appgree} ranks proposals by sampling voters in cycles, each with fewer proposals. This is
adequate for a range of decision making cases and showcases statistical estimates utility. The system has a separate proposition phase, and relies on an organized group engagement and user identities, which also does not fit current needs. 

Liquid Feedback~\cite{liquid} is a very renowned and bleeding edge solution for collective decision making.
It relies on delegating your voting count on specific subjects to other people you know or trust.
Therefore, it does not fit current needs. Even so, this framework have precious considerations for our case, such as about ranking and presenting proposals to voters in the most useful ways.

A Brazilian solution, used in diverse software and specially important as the output of a nation-wide decision making need, is the Agora Algorithm~\cite{agora}. It presents a decision procedure in phases (agenda proposition, deliberations proposition and commenting, voting) with resolution outcomes.
Although coherent, this framework requires authentication and might need experimentation and tuning in order to be effective with more than dozens or a few hundreds of participants.

There is a number of other solutions for online collaborative prioritization, such as IdeaScale, Kidling, or any flavor of an Analytic Hierarchy Process (AHP). Authors hope to better formalize possible solutions (and found implementations), maybe through recommender systems theory~\cite{coursera}.

% Pairwise, LiquidFeedback, Agora Communs/Delibera/Agora Algorithm. Apgree.
% IdeaScale, Kidling, AHP Methodology, Priorization methods,

% 

\section{Discussion}
The above estimates are the best fit the researchers could deliver,
suitable for current needs and not found (yet) in literature.
The following questions should be addressed in near future:
\begin{itemize}
	\item Are there more adequate metrics for ranking proposals in the given setting?
    \item What are strong and weak aspects of the approach for collective recommendation?
    \item What thresholds will be the choice of community and will they be adjusted in time?
    \item Are there really no previous formalized model of this setting? If there is, what comparisons can we make on design, metrics and outcomes?
    \item To which extent will participation community and public managers legitimize this approach?
	\item What is the impact of this technological approach in public health care, social participation and the scientific community?
    \item To which extent society benefits from this continuous voting process? Is it worth the time spent by voters? How to evaluate this relation in terms of spent and gained resources?
\end{itemize}

Most importantly, this report is being delivered to the civil society and scientific community for consideration. Given the large number of possibilities for the collective ranking procedure, and the proliferation of solutions, research efforts might aim the organization of such procedures.

%\begin{materials}
%\begin{theorem}
%with $|Error|\leq C\, \delta | log\delta| $ where $C$ depends only
%on $\|\theta\|_{L^{\infty}}$ and $\|
%\nabla\varphi\|_{L^{\infty}}$.
%\end{theorem}

%\end{materials}

%----------------------------------------------------------------------------------------
%	APPENDICES (OPTIONAL)
%----------------------------------------------------------------------------------------

%\appendix
%An appendix without a title.
%
%\appendix[Erd\"os sectors]
%An appendix with a title.
%\begin{definition}
%	The Erd\"os Sectors S of the network N are defined as the three sectors provenient from the comperrisson of
%	N to an Erd\"os-Renyi network with the same number of nodes and adges.
%\begin{eqnarray}
%equations
%\end{eqnarray}
%\end{definition}


%----------------------------------------------------------------------------------------
%	ACKNOWLEDGEMENTS
%----------------------------------------------------------------------------------------

\begin{acknowledgments}
	Author is grateful to CNPq (process
	140860/2013-4, project 870336/1997-5), UNDP
	(contract 2013/00056, project BRA/12/018), SNAS/SGPR,
	and the Postgraduate Committee of the IFSC/USP.
	Thanks to the Brazilian social participation community for the conception and practice of this specific voting setting.
\end{acknowledgments}

%----------------------------------------------------------------------------------------
%	BIBLIOGRAPHY
%----------------------------------------------------------------------------------------

%% PNAS does not support submission of supporting .tex files such as BibTeX.
%% Instead all references must be included in the article .tex document. 
%% If you currently use BibTeX, your bibliography is formed because the 
%% command \verb+\bibliography{}+ brings the <filename>.bbl file into your
%% .tex document. To conform to PNAS requirements, copy the reference listings
%% from your .bbl file and add them to the article .tex file, using the
%% bibliography environment described above.  

%%  Contact pnas@nas.edu if you need assistance with your
%%  bibliography.

% Sample bibliography item in PNAS format:
%% \bibitem{in-text reference} comma-separated author names up to 5,
%% for more than 5 authors use first author last name et al. (year published)
%% article title  {\it Journal Name} volume #: start page-end page.
%% ie,
% \bibitem{Neuhaus} Neuhaus J-M, Sitcher L, Meins F, Jr, Boller T (1991) 
% A short C-terminal sequence is necessary and sufficient for the
% targeting of chitinases to the plant vacuole. 
% {\it Proc Natl Acad Sci USA} 88:10362-10366.


%% Enter the largest bibliography number in the facing curly brackets
%% following \begin{thebibliography}

\begin{thebibliography}{10}

\bibitem{issue1}
Issue \#19 for implementing priorization algorithm. \url{https://gitlab.com/participa/proposal-app/issues/19}

\bibitem{issue2}
Issue \#29 for implementing priorization algorithm. \url{https://gitlab.com/participa/proposal-app/issues/29}

\bibitem{tabRonald}
Table developed for discussions about approval and participation indexes. Ronald Costa. \url{https://docs.google.com/spreadsheets/d/1Q2uXcuKYOuL0nJdpjXoUaQhjmj6mNv2St7F-pBi_3lw}

\bibitem{pairwise}
  Wiki surveys: Open and quantifiable social data collection,
  Salganik, Matthew J and Levy, Karen EC,
  arXiv preprint arXiv:1202.0500,
  \url{http://arxiv.org/abs/1202.0500}

\bibitem{appgree}
  Appgree software for collective ranking of proposals. \url{http://www.appgree.com}

\bibitem{liquid}
  Through Liquid Democracy to Sustainable Non-Bureaucratic Government,
  Paulin, Alois,
  CeDEM 2014,
  p. 205--217,
  2014, \url{http://research.apaulin.com/research/2014/cedem14-ld-snbg}


\bibitem{agora}
Agora Algorithm, Brazilian social participation community, \url{https://vimeo.com/39135943}

\bibitem{coursera}
Konstan, Joseph A., et al. "Teaching recommender systems at large scale: evaluation and lessons learned from a hybrid MOOC." Proceedings of the first ACM conference on Learning@ scale conference. ACM, 2014. \url{http://dl.acm.org/citation.cfm?id=2566244}

%\bibitem{Versinus}
%Fabbri, Renato. "Versinus: a visualization method for graphs in evolution." arXiv preprint arXiv:1412.7311 (2014).
%
%\bibitem{musicArticle}
%Fabbri, Renato, et al. "Psychophysics of musical elements in the discrete-time representation of sound." arXiv preprint arXiv:1412.6853 (2014).
%\bibitem{BN}
%M.~Belkin and P.~Niyogi, {\em Using manifold structure for partially
%  labelled classification}, Advances in NIPS, 15 (2003).
%
%\bibitem{BBG:EmbeddingRiemannianManifoldHeatKernel}
%P.~B\'erard, G.~Besson, and S.~Gallot, {\em Embedding {R}iemannian
%  manifolds by their heat kernel}, Geom. and Fun. Anal., 4 (1994),
%  pp.~374--398.
%
%\bibitem{CLAcha1}
%R.R.~Coifman and S.~Lafon, {\em Diffusion maps}, Appl. Comp. Harm. Anal.,
%  21 (2006), pp.~5--30.
%
%\bibitem{DiffusionPNAS}
%R.R.~Coifman, S.~Lafon, A.~Lee, M.~Maggioni, B.~Nadler, F.~Warner, and
%  S.~Zucker, {\em Geometric diffusions as a tool for harmonic analysis and
%  structure definition of data. {P}art {I}: Diffusion maps}, Proc. of Nat.
%  Acad. Sci.,  (2005), pp.~7426--7431.
%
%\bibitem{Clementi:LowDimensionaFreeEnergyLandscapesProteinFolding}
%P.~Das, M.~Moll, H.~Stamati, L.~Kavraki, and C.~Clementi, {\em
%  Low-dimensional, free-energy landscapes of protein-folding reactions by
%  nonlinear dimensionality reduction}, P.N.A.S., 103 (2006), pp.~9885--9890.
%
%\bibitem{DoGri}
%D.~Donoho and C.~Grimes, {\em Hessian eigenmaps: new locally linear
%  embedding techniques for high-dimensional data}, Proceedings of the National
%  Academy of Sciences, 100 (2003), pp.~5591--5596.
%
%\bibitem{DoGri:WhenDoesIsoMap}
%D.~L. Donoho and C.~Grimes, {\em When does isomap recover natural
%  parameterization of families of articulated images?}, Tech. Report Tech. Rep.
%  2002-27, Department of Statistics, Stanford University, August 2002.
%
%\bibitem{GruterWidman:GreenFunction}
%M.~Gr\"uter and K.-O. Widman, {\em The {G}reen function for uniformly
%  elliptic equations}, Man. Math., 37 (1982), pp.~303--342.
%
%\bibitem{Simon:NeumannEssentialSpectrum}
%R.~Hempel, L.~Seco, and B.~Simon, {\em The essential spectrum of neumann
%  laplacians on some bounded singular domains}, 1991.
%
%\bibitem{1}
%Kadison, R.\ V.\ and Singer, I.\ M.\ (1959)
%Extensions of pure states, {\it Amer.\ J.\ Math.\ \bf
%81}, 383-400.
%
%\bibitem{2}
%Anderson, J.\ (1981) A conjecture concerning the pure states of
%$B(H)$ and a related theorem. in {\it Topics in Modern Operator
%Theory}, Birkha\"user, pp.\ 27-43.
%
%\bibitem{3}
%Anderson, J.\ (1979) Extreme points in sets of
%positive linear maps on $B(H)$. {\it J.\ Funct.\
%Anal.\
%\bf 31}, 195-217.
%
%\bibitem{4}
%Anderson, J.\ (1979) Pathology in the Calkin algebra. {\it J.\
%Operator Theory \bf 2}, 159-167.
%
%\bibitem{5}
%Johnson, B.\ E.\ and Parrott, S.\ K.\ (1972) Operators commuting
%with a von Neumann algebra modulo the set of compact operators.
%{\it J.\ Funct.\ Anal.\ \bf 11}, 39-61.
%
%\bibitem{6}
%Akemann, C.\ and Weaver, N.\ (2004) Consistency of a
%counterexample to Naimark's problem. {\it Proc.\ Nat.\ Acad.\
%Sci.\ USA \bf 101}, 7522-7525.
%
%\bibitem{TSL}
%J.~Tenenbaum, V.~de~Silva, and J.~Langford, {\em A global geometric
%  framework for nonlinear dimensionality reduction}, Science, 290 (2000),
%  pp.~2319--2323.
%
%\bibitem{ZhaZha}
%Z.~Zhang and H.~Zha, {\em Principal manifolds and nonlinear dimension
%  reduction via local tangent space alignement}, Tech. Report CSE-02-019,
%  Department of computer science and engineering, Pennsylvania State
%  University, 2002.
\end{thebibliography}

%----------------------------------------------------------------------------------------

\end{article}

%----------------------------------------------------------------------------------------
%	FIGURES AND TABLES
%----------------------------------------------------------------------------------------

%% Adding Figure and Table References
%% Be sure to add figures and tables after \end{article}
%% and before \end{document}

%% For figures, put the caption below the illustration.
%%
%% \begin{figure}
%% \caption{Almost Sharp Front}\label{afoto}
%% \end{figure}

%\begin{figure}[h]
%\centerline{\includegraphics[width=0.4\linewidth]{placeholder.jpg}}
%\caption{Figure caption}\label{placeholder}
%\end{figure}

%% For Tables, put caption above table
%%
%% Table caption should start with a capital letter, continue with lower case
%% and not have a period at the end
%% Using @{\vrule height ?? depth ?? width0pt} in the tabular preamble will
%% keep that much space between every line in the table.

%% \begin{table}
%% \caption{Repeat length of longer allele by age of onset class}
%% \begin{tabular}{@{\vrule height 10.5pt depth4pt  width0pt}lrcccc}
%% table text
%% \end{tabular}
%% \end{table}

%\begin{table}[h]
%\caption{Table caption}\label{sampletable}
%\begin{tabular}{l l l}
%\hline
%\textbf{Treatments} & \textbf{Response 1} & \textbf{Response 2}\\
%\hline
%Treatment 1 & 0.0003262 & 0.562 \\
%Treatment 2 & 0.0015681 & 0.910 \\
%Treatment 3 & 0.0009271 & 0.296 \\
%\hline
%\end{tabular}
%\end{table}

%% For two column figures and tables, use the following:

%% \begin{figure*}
%% \caption{Almost Sharp Front}\label{afoto}
%% \end{figure*}

%% \begin{table*}
%% \caption{Repeat length of longer allele by age of onset class}
%% \begin{tabular}{ccc}
%% table text
%% \end{tabular}
%% \end{table*}

%----------------------------------------------------------------------------------------

\end{document}
